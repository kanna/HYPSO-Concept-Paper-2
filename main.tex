\documentclass[letterpaper, 10 pt, conference]{ieeeconf}
\IEEEoverridecommandlockouts    
\overrideIEEEmargins 
\usepackage{fancyhdr}
\usepackage{graphics} % for pdf, bitmapped graphics files
\usepackage{amsmath}
\usepackage{amssymb}
\usepackage{graphicx}
\usepackage{float}
\usepackage{hyperref}
\usepackage{flushend}
\usepackage{color,soul}
\usepackage{todonotes}
\usepackage{mathtools}
\DeclarePairedDelimiter\ceil{\lceil}{\rceil}
\pagestyle{plain}

\begin{document}
% \title{Ocean Color Remote Sensing using a COTS-assembled Hyperspectral Imager for the HYPSO-1 Mission}
\title{Ocean Color Hyperspectral Remote Sensing with High Resolution and Low Latency – the HYPSO-1 Mission}
%\title{Ocean Color Remote Sensing with HYPSO Mission: a Hyperspectral Imaging SmallSatellite}
%\title{Coordinated Optical Oceanographic Observations with Space, Aerial, Surface and Underwater Robotic Vehicles}

% \author{Mariusz E. Gr{\o}tte$^1$, Roger Birkeland$^2$, Evelyn Honor{\'e}-Livermore$^2$, Sivert Bakken$^1$, Joseph L. Garrett$^1$, \\ 
% Elizabeth F. Prentice$^1$, Dennis D. Langer$^1$, Alberto Dallolio$^1$, Jo\~{a}o F. Fortuna$^{1,3}$, Gara Quintana-D{\'i}az$^2$, \\ 
% Milica Orlandic$^2$, Amund Gjersvik$^2$, Marie B{\o}e Henriksen$^1$, Arnoldas Pečiukevičius$^4$, Rimantas Žičkus$^4$, \\ 
% Ernestas Kalabuckas$^4$, Žilvinas Kvedaravičius$^4$, J. Tommy Gravdahl$^1$, Harald Martens$^3$, Fred Sigernes$^5$, \\ 
% Geir Johnsen$^{5,6}$, Fernando Aguado-Aguelet$^{1,7}$, Cecilia Haskins$^8$, Annette Stahl$^1$, Nils Torbjörn Ekman$^2$, \\ 
% Egil Eide$^2$, Kanna Rajan$^{1,9}$, Tor A. Johansen$^1$
\author{Mariusz E. Gr{\o}tte$^1$, Roger Birkeland$^2$, Sivert Bakken$^1$, Joseph L. Garrett$^1$, Evelyn Honor{\'e}-Livermore$^2$ \\ 
Elizabeth F. Prentice$^1$, Fred Sigernes$^{1,3}$, Milica Orlandic$^{2}$, J. Tommy Gravdahl$^{1}$, Kanna Rajan$^{4}$, Tor A. Johansen$^1$
  \thanks{ $^1$Center for Autonomous Marine Operations and Systems
    (AMOS), Department of Engineering Cybernetics, Norwegian
    University of Science and Technology, Trondheim, Norway.}
  \thanks{
    $^2$Department of Electronic Systems, Norwegian University of Science and Technology, Trondheim, Norway.}
  \thanks{
    $^3$ University Center in Svalbard, Longyearbyen, Norway.}
		\thanks{
    $^4$ Faculty of Engineering, University of Porto, Porto, Portugal.}
  \thanks{Corresponding author: {mariusz.eivind.grotte@ntnu.no}}
}

%   \thanks{ $^1$Center for Autonomous Marine Operations and Systems
%     (AMOS), Department of Engineering Cybernetics, Norwegian
%     University of Science and Technology, Trondheim, Norway.}
%   \thanks{
%     $^2$Department of Electronic Systems, Norwegian University of Science and Technology, Trondheim, Norway.}
% 	\thanks{
%     $^3$IDLETechs AS, Trondheim, Norway.}
%       \thanks{
%     $^4$ NanoAvionics, Vilnius, Lithuania.}
%   \thanks{
%     $^4$ University Center in Svalbard, Longyearbyen, Norway.}
% 		\thanks{
%     $^6$ Department of Biology, Norwegian University of Science and Technology, Trondheim, Norway.}
%     		\thanks{
%     $^7$ Universidad de Vigo, Vigo, Spain.}
%     		\thanks{
%     $^8$ Department of Mechanical and Industrial Engineering, Norwegian University of Science and Technology, Trondheim, Norway.}
% 		\thanks{
%     $^9$ Faculty of Engineering, University of Porto, Porto, Portugal.}
%   \thanks{Corresponding author: {mariusz.eivind.grotte@ntnu.no}}
% }

\maketitle
\begin{abstract}
Ocean color processes with characteristic spectral information, such as algal blooms, demand consistent monitoring at lower costs and high-resolution remote sensing data that are quickly delivered after first observation. Given recent advances in sensor and microcomputer technology, we present the mission design of HYPSO-1, a 6U CubeSat at $500 \hspace{3pt} \rm{km}$ altitude in Sun-Synchronous Orbit hosting a COTS-built pushbroom hyperspectral imager. The flight-ready camera covers wavelengths in the visual and near-infrared range, has a spectral bandpass of approximately $3.33 \hspace{3pt} \rm{nm}$ and swath width of $70 \hspace{3pt} \rm{km}$. Since spatial resolution can be poor due to its small optics and high orbital altitude and speed, using fundamental principles in geometry we show how HYPSO-1's ability to perform a slew maneuver during imaging enables spatial resolution to become better than $100 \hspace{3pt} \rm{m}$. The imager's Signal-to-Noise Ratio when observing typical water-leaving radiance is also characterized. To allow efficient downlink of large hyperspectral datasets over limited radio communications and ground station passes, we have carefully designed FPGA-based on-board image processing pipelines that reduce data size with lossless compression and dimensionality reduction or by extracting only characteristic features with target detection or classification algorithms. We justify the concept of operations with a simulated scenario where HYPSO-1 first observes a $40 \hspace{3pt} \rm{km}$ by $40 \hspace{3pt} \rm{km}$ area in the coast of Lofoten, Norway, then downlinks various data products to selected ground stations. The data products compressed with CCSDS123v1 can be downloaded in less than $1 \hspace{3pt} \rm{hr}$ and $36 \hspace{3pt} \rm{min}$ when taking into account the overhead in internal spacecraft bus communications, and less than $10 \hspace{3pt} \rm{min}$ without. Using dimensionality reduction, target detection and classification, the data products have latency of just a few minutes. After launch, HYPSO-1 will determine efficacy in providing tailored high-resolution and low-latency ocean color data from hyperspectral imaging small-satellites.

% enable utilizing hyperspectral imagers on small-satellites at lower costs. 

 
% We also estimate the corresponding Signal-to-Noise Ratio per pixel for a typical water-leaving radiance.
% The HYPSO-1 mission will demonstrate the efficacy of COTS-based hyperspectral imaging in providing high-resolution data with low latency.

% \textcolor{blue}{
% Colorful ocean processes with distinguishable spectral characteristics, such as algal blooms, demands consistent monitoring and high-resolution remote sensing data from specific target areas that are quickly delivered after detection. 
% % To lower the cost and development time, cameras dedicated for ocean color such as hyperspectral imagers may be implemented on small-satellites. 
% We present a COTS-built pushbroom hyperspectral imager that is integrated on HYPSO-1, a 6U CubeSat at $500 \hspace{3pt} \rm{km}$ altitude in a Sun-Synchronous Orbit. The imager offers approximately $3.33 \hspace{3pt} \rm{nm}$ spectral resolution in the wavelength range of $271-1007 \hspace{3pt} \rm{nm}$ and a swath width of $70 \hspace{3pt} \rm{km}$ 
% % and approximately $58.6 \hspace{3pt} \rm{m}$ spatial resolution in the cross-track direction, although due to the satellite's speed the spatial resolution in the along-track direction can be worse than $500 \hspace{3pt} \rm{m}$ per frame. 
% but the spatial resolution is limited due to the camera's limitations on frame rate, high altitude and the satellite's speed at $7.61 \hspace{3pt} \rm{km/s}$. With a precise Attitude and Determination Control System, HYPSO-1 may improve the spatial resolution by performing a slew maneuver while imaging thus enabling the distance between sequential frames to become less than $100 \hspace{3pt} \rm{m}$ in the along-track direction and additionally the Signal-to-Noise Ratio is increased by the resulting overlapping frames. Such a slew maneuver requires more time for imaging thereby more frames are collected, rendering a datacube of a large size. To download such data over a limited radio link, HYPSO-1's important capability lies in the on-board FPGA-based image processing algorithms that aim for both alleviating the power budget and reducing the data with lossless compression and dimensionality reduction or by extracting only important features with target detection or classification. We present a simulated scenario where HYPSO-1 is observing a $40 \hspace{3pt} \rm{km}$ by $40 \hspace{3pt} \rm{km}$ target area in Lofoten, Norway, and downlinking data products to a selected ground stations at NTNU and KSAT Lite. It is shown that the data products that are reduced minimally with only CCSDS123v1, can be downloaded to the end user in less than $1 \hspace{3pt} \rm{hr}$ and $36 \hspace{3pt} \rm{min}$ when taking into account the overhead in internal spacecraft bus communications, and less than $10 \hspace{3pt} \rm{min}$ without in a best case. For data products that have undergone dimensionality reduction, target detection and classification, the data latency is just a couple of minutes. The HYPSO-1 mission will demonstrate the efficacy of COTS-based hyperspectral imaging in providing high-resolution data with low latency.}
\end{abstract}
\textit{Keywords: HYPSO-1; hyperspectral remote sensing; ocean color; space optics; slew maneuver; spatial resolution; on-board image processing; data latency}
\\
 \section{Introduction} \label{sec:intro}
% As a sink for green-house gases and as the environment for marine life and resources, the oceans' role and evolving state, is undeniable. 
% The influence of the changing climate and its impact on the Earth, where 70\% is covered by water, needs to be studied from several perspectives. This ranges from the fine scale, i.e. micro-biology, to the larger scale, like atmospheric phenomenon such as hurricanes, the extent of the global ice melt, and algal blooms. A variety of these phenomena can be detected from space by optical or radar measurements. 
Optical remote sensing is normally used in the context of observing colorful processes with large spatio-temporal extent such as algal blooms. A primary light-absorbing substance in the oceans is chlorophyll, which is involved in phytoplankton photosynthesis and provides clear water surface signatures \cite{Geir2011}. Other substances particularly those composed of colored dissolved organic matter (CDOM) and suspended matter (SM) mix in the turbid waters and also have distinguishable optical characteristics. In particular, colorful algal blooms have a significant impact on the coastal environment, marine life and the society. These sporadically appear worldwide \cite{jessup09} with their size ranging from tens to hundreds of square kilometers and have varying biomass concentrations. Harmful Algal Blooms (HABs) may cause great damage to sustainable human food sources, especially the seafood industry such as fisheries and aquaculture. Given an early warning, the damage can be mitigated if the blooms are detected in due time \emph{before} reaching the fish pens. 
% (for example, operators may increase the amount of oxygen at the bottom of the fish pens or physically moving them away). 

Hyperspectral imaging is a promising method for detecting and monitoring algal blooms, and can create informative data with hundreds of narrow spectral bands \cite{Kutser2006}. Phytoplankton coloration is highly variable and often categorized as “red tides”, “green tides” or “brown tides” in the wavelength range of $400-700 \hspace{3pt} \rm{nm}$ \cite{Kutser2006, Johnsen1997, Geir2011, IOCCG2014}. Since the absorption spectra of substances are characteristic, hyperspectral imaging makes it possible to identify the primary production of algal blooms and also the overall health of the ocean based on its physical and chemical composition. According to \cite{IOCCG2014}, plankton and algae types or even species may be directly distinguished or inferred by the correlations with geo-physical parameters. For example, hyperspectral data may reveal the subtle spectral inflections imparted by specific pigment complements. With high spectral resolution, the volume of obtainable information may be increased by a four-figure factor as compared to multi-spectral data products \cite{Ortenberg2011}. Important information about aerosols and water vapor in the sensor's optical path through the atmosphere may be retrieved as well. The highly informational and operational characteristics would enhance the solution level of space monitoring tasks \cite{Villafranca2012}. However, keeping in mind the constraints imposed by the physics of optics on the signal-to-noise ratio (SNR), it is also challenging to downlink the large amounts of data generated by such missions \cite{guelman2009small}. 

Earth observation (EO) satellites operated by agencies such as The National Aeronautics and Space Administration (NASA)  and The European Space Agency (ESA) provide excellent data that cover the Earth on a global scale \cite{knight2014,Aguirre2007}. These have medium to high spatial resolution, but have low spectral resolution and several days before revisiting selected targets \cite{Ack16}. Furthermore, algal blooms may develop or disappear faster than the revisit time of a single satellite, and the data requires careful analysis, processing and validation which usually results in a slow delivery of data products to the end users and public. For example, Sentinel-3’s tailored multispectral sensor and temporal resolution as a stand-alone system is insufficient for monitoring algal blooms without using traditional and often time-consuming methods to retrieve important information \cite{Ogashawara2019}. On the other hand, hyperspectral remote sensing missions are many \cite{Gue16, Sou16, dierssen2015space, Guanter2015, 2014RemS66790M, Keith14, guelman2009small, pearlman2003}, and large satellite EO missions such as the Plankton, Aerosol, Cloud, Ocean Ecosystem (PACE) mission show great promise for utilizing hyperspectral imaging in ocean color applications in general \cite{Werdell2019}.  A small-satellite, often categorized as nano- and micro-satellite \cite{Gue16}, have a shorter lifetime than a large satellite, but can frequently be replaced with updated technology and have lower development and production costs \cite{modern_small_satellites}. 

Single-purpose small-satellites that are suitable for hyperspectral imaging may provide high spatial resolution by observing dedicated target areas of interest. Choosing to observe smaller target areas enables dedicated overlay of each satellite's image. Even though small cameras are limited in spatial resolution when used in Low-Earth-Orbit (LEO), the combination of precise Attitude Determination \& Control System (ADCS) and high camera frame rate may increase the spatial resolution in the images by ensuring sufficient number of overlapping frames throughout the observation. Small-satellites effectively serve as complementary platforms to Autonomous Underwater Vehicles (AUVs), Unmanned Surface Vehicles (USVs), Unmanned Aerial Vehicles (UAVs) and buoys which are limited in mobility and speed \cite{Dic05}. 

With the need for more data containing high resolution and accuracy from specific coastal regions, we present a concept of operations for an upcoming small-satellite mission developed at The Norwegian University of Science and Technology (NTNU), named HYPSO-1 which is a 6U CubeSat. The contributions in this paper are (a) the design of a COTS-assembled pushbroom hyperspectral imager used in the HYPSO-1 mission; (b) the concept of HYPSO-1's remote sensing strategy which enhances the spatial resolution and SNR in the hyperspectral data by performing a steady slew maneuver throughout image acquisition; and (c) We present HYPSO-1's on-board image processing pipeline that mainly aims to reduce data latency between image acquisition and end user, provide high-resolution tailored data products, and consequently alleviate the satellite's power budget. 

This paper is organized as follows. Section \ref{sec:mission-design} describes the ocean color requirements that motivate the choice of imager technology, the HYPSO-1 Concept of Operations (CONOPS) and some key camera and spacecraft capabilities. Section \ref{sec:payload-hsi} presents the design, physics and performance of the chosen pushbroom hyperspectral imager. Section \ref{sec:sampling} describes the remote sensing strategy of how the hyperspectral imager shall be used in practice along with the expected results from numerical simulations. In Section \ref{hypso-mission}, we present HYPSO-1 spacecraft bus and image processing pipeline and justifying the feasibility of the concept based on its subsystems, power budget and analysis on data latency for different chosen camera modes and image processing levels. Finally, we conclude our findings in Section \ref{sec:conclusions}.
\section{Mission Design} \label{missiondesign}
Since the mission is highly oriented towards science and remote sensing, as well as being a technology demonstrator, it is necessary to establish \emph{what} is needed and \emph{how} the mission will be conducted operationally to achieve full success criteria. Mission design study, or Phase A, motivate the particular use of HSI for ocean color purposes, and establishes how \hypso shall operate. Mission objectives, success criteria, requirements and constraints are established in Pre-phase A followed by analysis and characterization analysis for orbit, payload and operations in Phase A.

The \hypso mission objectives are as follows:
\begin{enumerate}
\item To provide and support ocean color mapping through a Hyperspectral Imager (HSI) payload, autonomously processed data, and on-demand autonomous communications in a concert of robotic agents at the Norwegian coast.
\item To collect ocean color data and to detect and characterize spatial extent of algal blooms, measure primary productivity using emittance from fluorescence-generating micro-organisms, and other substances resulting from aquatic habitats and pollution to support environmental monitoring, climate research and marine resource management.
\item Develop robust framework for rapid systems engineering for a pipeline of spacecraft that may optimize project development in academia and industry.
\item Build strong competence and strengthen the prospect of nano- and micro-satellite systems as supporting intelligent agents in integrated autonomous robotic systems dedicated to marine and maritime applications in Norway and internationally, these being applicable to communications and remote sensing (altimetry, SAR, radiometry etc.).
\item Describe scientific methodology that will be adopted for the research, and coordinate the project plans with other ongoing research activities at NTNU and other research institutions and companies.
\end{enumerate}

Furthermore, it is emphasized that this mission is developed by PhD students, researchers, Master's students and professors, hence it shall be of academic nature and include objectives to emanate publishable results in the respective domains of control theory, artificial intelligence, electrical engineering, aerospace engineering, marine technology, biology and remote sensing. 

Given the objectives it is important to establish a Level-0 statement for the mission:\\

\textbf{\hypso mission shall, through narrow field-of-view push-broom Hyperspectral Imaging, demonstrate proof-of-concept oceanographic observations dedicated to ocean color remote sensing by intelligently supporting a concert of robotic agents consisting of UAVs, USVs, AUVs and stationary buoys observing the same target areas.}\\

\subsection{Mission Architectures}
During Pre-Phase A, by listing trade-offs for each mission element one may construct architectures that differ in the properties thus having varying impacts on cost, design and operations as a function of design solutions. The most important mission elements are listed as follows: A = Mission Concept; B = Controllable Subjects; C = Passive Subject; D = Payload; E = S/C bus; F = Orbit; G = Launch; H = Ground System; I = Communications Architecture; J = Mission Operations. Table \ref{tab:mission_elem_opts} shows the options for each mission element selected. Baseline elements are indicated with number 1 and \textcolor{orange}{orange} text signifies alternatives to baseline solution.
\begin{table}[htbp]
	\caption{Mission Architectures}
	\label{tab:mission_elem_opts}
	\centering
		\begin{tabular}{|p{0.8cm}|p{7cm}|}
			\hline
		\textbf{Element}		&		\textbf{Option}			\\ 
			\hline 															  
			A1 &  HSI mapping of the ocean; autonomous onboard processing of mission data, then transmitted after pass; ground commands on mission plan.  \\
			\hline
			A2 & HSI mapping of the ocean; autonomous onboard processing of mission data, then transmitted after pass; ground commands on mission plan; \textcolor{orange}{updates to other robotic agents}. \\
			\hline
			A3  & HSI mapping of the ocean; \textcolor{orange}{semi-raw downlinked mission data, then post-processed}; ground commands on mission plan.  \\
			\hline
			A4  & HSI mapping of the ocean; \textcolor{orange}{semi-raw downlinked mission data, then post-processed}; ground commands on mission plan; \textcolor{orange}{if satellite sees interesting signature $\rightarrow$ send out other air/surface agents directly}  \\
			\hline	
			A5  & HSI mapping of the ocean; autonomous onboard processing of mission data, then transmitted after pass; ground commanding on mission plan; \textcolor{orange}{autonomous coordinated robotic multi-agent observations}   \\ \hline
			B1 & No agents tracked from space  \\ \hline
			B2  & \textcolor{orange}{Multi-agent targets tracked: USVs, UAVs, Ships, Buoys} \\ \hline
			C1  & Oceanography through Hyperspectral imaging    \\ \hline	
			D1  & Small aperture HSI \\ \hline
			D2  & SDR \\ \hline
			E1 & 2-6U size; 3-axis stabilization; spacecraft pointing; body-mounted solar panels; onboard GPS; onboard orbit control; no micro-propulsion \\ \hline
			F1 & SSO; 1-satellite \\ \hline
			F2 & \textcolor{orange}{(P)LEO}; 1-satellite \\ \hline
			G1 & PSLV or Soyuz 9 (highly tradeable) \\ \hline
			H1 & Dedicated: NTNU; Commercial (e.g. KSAT) \\ \hline
			I1 & Store \& dump data; TM/TC-transceiver; $\geq$2 ground stations; UHF-band uplink, X-band downlink \\ \hline
			I2 & Store \& dump data; TM/TC-transceiver; $\geq$2 ground stations; UHF-band uplink, \textcolor{orange}{S-band downlink} \\ \hline
			I3 & Store \& dump data; TM/TC-transceiver; $\geq$2 ground stations; \textcolor{orange}{S-band uplink}, X-band downlink \\ \hline
			I4 & Store \& dump data; TM/TC-transceiver; $\geq$2 ground stations; \textcolor{orange}{S-band uplink}, \textcolor{orange}{S-band downlink} \\ \hline
			I5 & Store \& dump data; TM/TC-transceiver; $\geq$2 ground stations; UHF-band uplink, X-band downlink; \textcolor{orange}{multi-agent cross-links in VHF/UHF} \\ \hline
			I6 & Store \& dump data; TM/TC-transceiver; $\geq$2 ground stations; UHF-band uplink, S-band downlink; multi-agent cross-links in VHF/UHF \\ \hline
			I7 & Store \& dump data; TM/TC-transceiver; $\geq$2 ground stations; \textcolor{orange}{S-band uplink}, X-band downlink; \textcolor{orange}{multi-agent cross-links in VHF/UHF} \\ \hline
			I8 & Store \& dump data; TM/TC-transceiver; $\geq$2 ground stations; \textcolor{orange}{S-band uplink}, \textcolor{orange}{S-band downlink}; \textcolor{orange}{multi-agent cross-links in VHF/UHF} \\ \hline
			J1 & Fully automated ground stations; part-time operations on demand; Indirect updates on mission to/from other agents
\\ \hline
			J2 & Fully automated ground stations; part-time operations on demand; \textcolor{orange}{Direct updates on mission to/from other agents} \\ \hline
		\end{tabular}
\end{table}

All these combinations of mission elements give 40 mission architectures. The elements are highly correlated, therefore justifying the need for detailed unbiased decision criteria analysis. System drivers such as program cost, risk, mission reliability, development reliability, man-hours, science output and size of data rate are weighted highest across a normalized scale. The drivers and mission elements are fed into a black box, or Decision-Making Analysis, i.e. \emph{TOPSIS} or \emph{AHP} as shown in Fig. \ref{fig:decision_making} \cite{Cascales2012, Saaty1987}. These "black-box" methods select the most reliable candidates of mission architectures, ranking them as shown in Table \ref{tab:topsis}. The top 5 architectures are chosen to investigate here and will be iterated on further until choosing only one with confidence for detailed systems design.
\begin{figure}[htbp]
  \begin{center}
    \includegraphics[width=85mm,angle=0]{figs/decision_making.png}
    \caption{Input to decision making 1st order or multivariable algorithms with rank driven by system drivers}
    \label{fig:decision_making}
\end{center}
\end{figure}
\begin{table}[htbp]
	\caption{TOPSIS Decision Ranking}
	\label{tab:topsis}
	\centering
		\begin{tabular}{|p{1.1cm}|p{5cm}|}
			\hline
		\textbf{Rank}	&		\textbf{Mission Architecture}	\\ 
			\hline 															  
			1 & A2-B1-C1-D1-E1-F1-G1-H1-I8-J1  \\
			\hline
			2 & A1-B1-C1-D1-E1-F2-G1-H1-I2-J1 \\
			\hline
			3  & A3-B1-C1-D1-E1-F2-G1-H1-I1-J1  \\
			\hline
			4  & A3-B1-C1-D1-E1-F1-G1-H1-I2-J1  \\
			\hline	
			5  & A2-B2-C1-D1-E1-F1-G1-H1-I6-J2  \\ \hline
		\end{tabular}
\end{table}
\subsection{Science Requirements}
\begin{table*}[htbp]
	\centering
			\caption{Available biology in Norway/Scandinavia}
		\begin{tabular}{llll}
			\hline
			Class & Color & Location & Season \\
			\hline
			Diatoms & Green/yellow & S to Mid-West Norway & Mar-Jun \\
			Prymnesiophytes & Golden/brown & All Norway & Apr-Jul \\
			Raphidophytes/Dictyochophytes & Golden/brown & South-West Norway & Apr-May \\
			Cyanophytes & Reddish & Baltic/Skagerrak/South Norway & Jul-Sep \\
			\hline
			Species (\redtext{red} = TOXIC) & Color & Location & Season \\
			\hline
			\textit{Skeletonema costatum} & Golden/brown & Skagerrak & May-Jun \\
			\textit{Chaetoceros convolutus} & Golden/brown & Rogaland-Helgeland & Mar-Apr \\
			\redtext{\textit{Prymnesium parvum}} & Golden & Hylsfjord in Ryfylke & Jul-Aug \\
			\redtext{\textit{Chrysochromulina polylepis}} & Brown & S, SE, W and Mid-Norway, Oster/S\o rfjord & Apr-Jul \\
			\redtext{\textit{P. papilliferum}} & Golden & Hylsfjord in Ryfylke & Jul-Aug \\
			\redtext{\textit{Heterosigma akashiwo}} &	Reddish & Osterfjord/S\o rfjord & Apr-May \\
			\redtext{\textit{Karenia mikimotoi}} & Golden/brown & Skagerrak/Baltic & Apr-Aug \\
			\redtext{\textit{Karlodinium veneficum}} & Golden/brown & Skagerrak/Baltic & Apr-Aug \\
			\textit{Emiliania huyxlei} & Milky/brown & Along all Norwegian Coast & Apr-Sep \\
			\textit{Pseudochattonella} & Golden/brown & Baltic & Apr-Aug \\
			\hline
		\end{tabular}
		\label{tab:biology}
\end{table*}
Since the main driver for this mission is oceanography, specifically dedicated to narrow field-of-view monitoring and mapping of ocean color phenomena particularly linked to biology, the key science objectives are:
\begin{itemize}
\item Detect algae and phytoplankton in coastal waters (see Table \ref{tab:biology} for relevant biology)
\item Enable $<$100 m pixel resolution and high spectral resolution of at least 10 nm to characterize useful signatures
\item Detect color of other matter such as biology, color-distorted organic matter, oil spills and river plumes
\item Distinguish harmful and non-harmful species cooperatively from space observations (inferral) and in-situ measurements (validation)
\item Enable remote sensing corrections for atmosphere, aerosols, air bubbles, sun-glint, water turbidity, diffracted second order light, water vapor, landscape distortions
\item In-situ validation of remote sensing data will be necessary by methods of using USVs, AUVs or manual sample collection
\item Space remote sensing shall be coordinated with NTNU AUV field campaigns in Svalbard, Trondheim and Fr{\o}ya
\item Positive detections of relevant signatures from space are to be investigated closer by UAV, USV or AUV with high response
\item Observations shall be available in Spring/Summer time from March to July when biology is relatively abundant and probability of detection is highest
\end{itemize}
One of the main phytoplankton classes that are common in Norwegian ocean are a) Diatoms; b) Prymnesiophytes; c) Raphidophytes/Dictyochophytes; d) and Cyanophytes aka Cyanobacteria  \cite{Geir2011}. Algae/plankton classes and species to look for in Norway/Scandinavia are listed in Table \ref{tab:biology}.

\subsection{Payload Requirements}
Some selected payload requirements needed to fulfill the mission requirements and science objectives are as follows:
\begin{itemize}
\item Faintest detectable ToA signature for on-board algorithm detection shall be at least SNR of 100:1 in the range of 400-600 nm range and at least SNR of 40:1 in the 600-800 nm range 
\item Onboard processing shall consist of automated geometric (situational awareness) processing/calibration; radiometric processing/calibration; spectral compression; and spatial compression in the respective order and have feedback loop to the navigational and control \& task execution data from ADCS
\item Corrections for atmospheric distortions, water particles, aerosols, turbidity, clouds shall be enabled by utilizing $750-800$ nm (NIR) bands
\item Four imaging modes shall be enabled: 1) high-resolution with 160 spectral bands; 2) medium-resolution with 160 spectral bands; 2) high-resolution with 16 spectral bands; 3) medium-resolution with 16 spectral bands
\item On-board super-resolution or deconvolution algorithms shall enable overlapping fields of view to be fused in order to enhance the image resolution by a factor of at least $3$ and mean SNR of at least $\sqrt{3}$
\item Level 2 data transmitted to ground shall consist of geometrically and radiometrically calibrated and geo-referenced hyperspectral images with up to 160 spectral bands and $\leq 10$ nm resolution that have Gaussian average for each band
\item Level 4 data transmitted to ground shall consist of target location and at least radiometrically calibrated hyperspectral images with up to 20 spectral bands and $\leq 5$ nm resolution that have Gaussian average for each band 
\item Payload shall operate in unique modes according to the database used (e.g. gain tuning, exposure time, binning operations, and spectral compression)
\item Payload shall enable on-board radiometric and geometric calibration resulting in $\leq 30$\% radiometric uncertainty and $\leq 10$ \% geometric uncertainty
\end{itemize}
\subsection{Orbit Selection}
\begin{figure*}[htbp]
  \begin{center}
    \includegraphics[width=130mm,angle=0]{figs/groundtrack1.png}
    \caption{Groundtrack \hypso in morning, evening and ISS orbits at epoch 16 May 2018 07:00:00 (UTC).}
    \label{fig:groundtrack1}
\end{center}
\end{figure*}
The orbit is selected given a preferred observation target in the coast of Mid-Norway and prospective Ground Stations in Trondheim, Troms{\o} and Svalbard for communications. Orbit parameters are summarized in Table \ref{tab:mission_params}.
\begin{table}[htbp]
	\caption{Baseline Orbit Configuration}
	\label{tab:mission_params}
	\centering
		\begin{tabular}{|l|l|}
			\hline
			Orbit Parameter			&	 Value 			\\ 
			\hline
			Launch LTAN &       10:00 AM/10:00 PM \\
			Semi-major axis, $a$ &  6878.14 km \\
			Altitude, $h$  &     500 km \\
			Average altitude loss & -3.4 m/day \\
			Speed, $v_{\text{sat}}$ & 7.621 km/s \\
			Orbit period & 94 min 49 s \\
			Inclination, $i$ &         97.31$^{\circ}$ \\
			Eccentricity, $e$ &       0.0015 \\
			RAAN precession rate & 3.6$\times 10^{-5 \hspace{3pt}\circ}$/day westwards \\
			Angular momentum & 52261.69 km$^2$/s \\
			Revolutions & 15.31 revs/day \\
			Repeat cycle & 7 days \\	
			Mean eclipse ratio & 36.1 \% \\
			Lifetime & 7.4 years \\
		 \hline
		\end{tabular}
\end{table}
A sun-synchronous orbit is chosen which is a near-polar orbit with inclination $i=96-98^{\circ}$ and altitude $h=500-600$ km. The advantage is that the satellite passes over any given point of the Earth's surface at the same local sidereal time, however J2 effects or oblateness of the Earth will precess the nominal RAAN, $\Omega$, but less as compared to a polar orbit. The orbit is chosen such that at least approximately $60\%$ of the orbit is in constant sunlight and other $40\%$ in Earth's shadow (Umbra) in order to meet the mission requirements to observe a target off the coast of central Norway during morning or mid-day.

\subsubsection{Targets} \label{sec:targets}
Baseline targets to be observed are: a) Fr{\o}ya; b) Barents Sea (North of Finnmark County); c) Baltic Sea; d) Lofoten; e) Azores \& Portugal; f) Monterey Bay; g) Lake Hudson; h) East Greenland; i) Svalbard. All of these regions have significant history of algal blooms and appearance of non-nominal ocean color and biology and are therefore of interest to observe. Specifically, Fr{\o}ya, Barents Sea, Lofoten and Svalbard are interesting to observe from space in order to support AUV field campaigns (sampling and underwater imaging) run by NTNU regularly \footnote{\hypso may, through mission control communications, aid AUVs by directing them towards corrected coordinates based on what the satellites sees which will significantly save both operational costs and time.}. 

\subsubsection{Orbit Configuration 1} \label{sec:orbit1}
Parameters for a morning SSO configuration are $h=500$ km, $i\approx 97.31^{\circ}$ and LTAN 10:00 AM at launch, and is called a "morning" orbit since the Right Ascension of Ascending Node (RAAN) crosses the equatorial plane in the morning 10:00 AM (LTAN). Figs. \ref{fig:groundtrack1} and \ref{fig:orbit_sso} show a SSO \sml configuration ground track and 3D view, respectively. Note that the Norwegian coast is covered several times per day (both northwards and southwards passes as Earth revolves about its axis), although the orbit track does not give the satellite flexibility in observing along the coast since it passes Norway cross-track to the coast. Ground track repeat cycle is about 7 days for this configuration.
%\begin{figure}[H]
  %\begin{center}
    %\includegraphics[width=85mm,angle=0]{figs/altitudevstime_sso.png}
    %\caption{Altitude changes during one day due to orbit not being completely circular.}
    %\label{fig:altitudevstime_sso}
%\end{center}
%\end{figure}
Details about access times to selected targets and Ground Stations are given in Table \ref{tab:revisit_1}, , where \textcolor{blue}{blue} indicates ground station and \textcolor{red}{red} indicates target to image. It is assumed that target areas and ground stations have elevation angles of $\epsilon_{\text{Target}}=20^{\circ}$ and $\epsilon_{\text{GS}}=10^{\circ}$ respectively where first is due to optical viewing angle constraints ($\theta<70^{\circ}$).
\begin{figure}[htbp]
  \begin{center}
    \includegraphics[width=75mm,angle=0]{figs/orbit_sso.png}
    \caption{Two possible \hypso orbits (SSO) at altitude $h=500$ km and ISS orbit.}
    \label{fig:orbit_sso}
\end{center}
\end{figure}
\begin{table}[htbp]
	\caption{Access times (16 May 2020 07 AM - 17 May 2020 07 AM) for Configuration 1}
	\label{tab:revisit_1}
	\centering
		\begin{tabular}{|l|c|c|c|c|c|}
			\hline
			 & \textcolor{blue}{NTNU} & \textcolor{blue}{Svalbard} & \textcolor{blue}{UPorto}  &  \textcolor{red}{Fr{\o}ya} & \textcolor{red}{Barents} \\
					\hline
			\# passes & 7 & 11 & 2 & 3 & 4 \\
			Max (min) & 7.408 & 7.478 & 7.446 & 5.011 & 5.517 \\
			Mean (min) 	&	5.478	&	6.813 & 7.315 & 3.397 & 3.997  		\\
			Min (min) & 2.780 & 4.743 & 7.185 & 1.888 & 0.928 \\
			\hline
		\end{tabular}
\end{table}
%Interesting observable locations and Ground Station functionality and availability are given in Table \ref{tab:mission_ops1}.
%\begin{table*}[htbp]
	%\caption{South-North Pass Observations on 22 June}
	%\label{tab:mission_ops1}
	%\centering
		%\begin{tabular}{|l|l|l|l|}
			%\hline
			%\textbf{\#}			&	 \textbf{Time (UTC)}	&	\textbf{Targets} 	& \textbf{Ground Stations}		\\ 
			%\hline 															
		 %1  &  08:09:00     &  Svalbard; Barents Sea; Trondheim (1st) & Svalbard (DOWN); Trondheim (UP) \\
		 %2  &  09:39:10     &  Trondheim (2nd); Fr{\o}ya; Lofoten; Baltics; Svalbard & Svalbard (DOWN); Trondheim (UP); Porto (DOWN/UP) \\
		 %3  &  10:51:20     & Iceland; Faroe Islands; South Africa; Trondheim (3rd); Ireland; UK & Svalbard (DOWN); Trondheim (UP); Porto (UP) \\
		 %4 & 12:40:00 & West Africa; Iceland; Greenland & None \\
		 %5 & 15:57:00 & Lake Hudson & None \\
		 %6 & 17:21:19 & Mexico Gulf & None \\
		 %7 & 18:59:00 & Monterey Bay, CA & NASA Ames \\
			%\hline
		%\end{tabular}
%\end{table*}
%Mission phases may be summarized in the following Table \ref{tab:mission_phases1}, assuming $20^{\circ}$ viewing angle for HSI observations.
%\begin{table}[htbp]
	%\caption{Mission Phases in Orbit 1 Concept}
	%\label{tab:mission_phases1}
	%\centering
		%\begin{tabular}{|l|l|l|r|}
			%\hline
			%Segment &		Description		& Start (UTC)	& Duration (s) \\
			%\hline
			%Phase 1 &	Harvest &	09:37:10 &	5 \\
			%Phase 2	& Comms. Trondheim	& 09:37:15 &	125 \\
			%Phase 3	& Prepare slewing	& 09:39:20	& 115 \\
			%Phase 4	& HSI operations	& 09:41:15 &	54 \\
			%Phase 5	& Data processing	& 09:42:09 &	74 \\
			%Phase 6	& Point to Svalbard &	09:43:25	& 20 \\
			%Phase 7	& Comms. Svalbard	& 09:43:45 & 270 \\
			%Phase 8	& Harvest & 09:48:15 &	605 \\
			%Phase 9	& Sleep	& 09:59:20 &	2207 \\
			%Phase 10 &	Harvest &	10:36:07	& 2255 \\
			%N+1 &	Next target	& 11:13:42	& 373 \\
			%\hline
			%\end{tabular}
%\end{table}
\subsubsection{Orbit Configuration 2} \label{sec:orbit2}
Second configuration has $h=500$ km, $i\approx 97.31^{\circ}$ and 10:00 PM LTAN, which is called an "evening" orbit. Figure \ref{fig:orbit_sso} shows the particular orbit configuration where the satellite, at this particular time of the year, goes from north to south. Note that the Norwegian coast is covered several times per day. One particular orbit may potentially pass all the way from Svalbard down to the tip of Southern Norway, covering the whole coast. Appropriate launch window must be chosen wisely to avoid sun glare effects though these are fewer than in morning. Since many biological events happen in the Spring around 10 AM, this orbit appears more scientifically viable as it offers flexibility along observing the whole coast. Ground track repeat cycle is about 7 days also for this configuration.

Details about access times to NTNU, Longyearbyen, UPorto, Fr{\o}ya and Barents are given in Table \ref{tab:revisit_2} where \textcolor{blue}{blue} indicates ground station and \textcolor{red}{red} indicates target to image.
\begin{table}[htbp]
	\caption{Access times (16 May 2020 07 AM - 17 May 2020 07 AM) for Configuration 2}
	\label{tab:revisit_2}
	\centering
		\begin{tabular}{|l|c|c|c|c|c|}
			\hline
			 & \textcolor{blue}{NTNU} & \textcolor{blue}{Svalbard} & \textcolor{blue}{UPorto}  &  \textcolor{red}{Fr{\o}ya} & \textcolor{red}{Barents} \\
					\hline
			\# passes & 6 & 11 & 4 & 2 & 5 \\
			Max (min) & 7.405 & 7.446 & 7.182 & 5.005 & 5.350 \\
			Mean (min) 	&	5.725	&	6.484 & 5.375 & 3.835 & 4.507  		\\
			Min (min) & 2.349 & 3.460 & 2.871 & 2.665 & 3.471 \\
			\hline
		\end{tabular}
\end{table}
%Interesting observable locations and Ground Station functionality and availability are given in Table \ref{tab:mission_ops2}.
%\begin{table*}[htbp]
	%\caption{North-South Pass Observations on 28 April 2020}
	%\label{tab:mission_ops2}
	%\centering
		%\begin{tabular}{|l|l|l|l|}
			%\hline
			%\textbf{\#}			&	 \textbf{Time (UTC)}	&	\textbf{Options} 	& \textbf{Ground Stations}		\\ 
			%\hline 															
		 %1  &  07:51:30     &  Barents Sea; South Africa & Svalbard (UP); Trondheim (DOWN) \\
		 %2  &  09:25:55     &  Svalbard; All Norway; Denmark & Svalbard (UP); Trondheim (DOWN) \\
		 %3  &  11:02:30     & Iceland; Faroe Islands; Azores (PT); Ireland; UK & Svalbard (UP); Porto (DOWN) \\
		 %4 & 15:46:10 & Lake Hudson & None \\
		 %5 & 17:26:00 & Monterey Bay & NASA Ames \\
			%\hline
		%\end{tabular}
%\end{table*}

\subsubsection{Backup Orbit} \label{sec:backup}
Constraints due to costs and budgets may have an impact on the project development and hence result in a less desirable but affordable orbit that are accessible with cheaper launches. For instance launches are most abundant and frequent to ISS due to high-demand for supply, maintenance and rapid science measurements on the space station. ISS is also at a lower altitude and low inclination of $i\approx 51.6^{\circ}$, hence launcher costs are lower. This backup orbit is also analyzed in case of an orbit insertion/deployment going wrong for a nominal SSO launch where it is stipulated in the worst case that \hypso will have a low inclination hence access not being granted to either NTNU, Svalbard or nominal target areas in Fr{\o}ya, Baltic and Barents Sea. Figures \ref{fig:groundtrack1} and \ref{orbit_sso.png} show the ground track and orbit, respectively. Details about access times to selected targets and Ground Stations are given in Table \ref{tab:revisit_1}, where \textcolor{blue}{blue} indicates ground station and \textcolor{red}{red} indicates target to image.
\begin{table}[htbp]
	\caption{Access times (16 May 2020 07 AM - 17 May 2020 07 AM) for ISS Configuration}
	\label{tab:revisit_2}
	\centering
		\begin{tabular}{|l|c|c|c|c|c|}
			\hline
			 & \textcolor{blue}{UPorto} & \textcolor{blue}{UVigo} & \textcolor{red}{Monterey} & \textcolor{red}{Azores} & \textcolor{red}{Cape Point} \\
					\hline
			\# passes & 7 & 7 & 1 & 1 & 3 \\
			Max (min) & 6.554 & 6.581 & 3.963 & 4.390 & 4.221 \\
			Mean (min) 	&	4.740	&	4.839 & - & - & 2.706 \\
			Min (min) & 3.340 & 2.723 & - & - & 0.558 \\
			\hline
		\end{tabular}
\end{table}


\section{Pushbroom Hyperspectral Imager} \label{sec:hsi}
\begin{figure*}[htbp]
  \centering
      \includegraphics[width=0.7\textwidth]{figs/optics.png}
  \caption{Optical diagram of the pushbroom hyperspectral imager based on \cite{Sigernes18}.}
	\label{fig:optics}
\end{figure*}
% \textcolor{blue}{Contribution (2): "We present a COTS-assembled hyperspectral imager with high spectral resolution and efficiency designed for the HYPSO-1 mission and is based on \cite{Sigernes18}. We present the chosen components for the design that satisfy objectives of observing algal blooms, and show how SNR may be increased by binning operations." Key points to keep in mind while writing this section:
% \begin{itemize}
%     \item How does the proposed HSI work?
%     \item How is it performing from space on HYPSO-1?
% \end{itemize}}
\subsection{Optics}\label{sec:optics}
Emerging commercial off the shelf (COTS) products that are smaller in size make hyperspectral imaging accessible, flexible and affordable \cite{Sigernes18}. Such low-cost pushbroom cameras may in principle be designed and integrated into small-satellites.
\begin{figure}[htbp]
  \centering
      \includegraphics[width=0.45\textwidth]{figs/pushbroom_scanning.png}
  \caption{Data acquisition with pushbroom hyperspectral imaging.}
	\label{fig:push_scan}
\end{figure}
% Figure \ref{fig:push_scan} shows a satellite-based pushbroom hyperspectral imager collecting lines of cross-track pixels and spectral information, eventually forming a datacube. 
Figure \ref{fig:optics} shows the optical diagram of a center cross-section of the instrument parallel to the refraction axis. The components are: (i) front lens with aperture diameter $D_0$ and focal length $F_0$; (ii) entrance slit with dimensions $w_{\text{slit}}$ and $h_{\text{slit}}$ that are slit width and height, respectively; (iii) collimator lens with aperture diameter $D_1$ and focal length $F_1$; (iv) grating that receives the incoming light at angle $\alpha=0^{\circ}$ and diffracts the light at angle $\beta$ measured from the grating normal; and (v) detector lens with aperture diameter $D_2$ and focal length $F_2$. Finally, (vi) is the detector. The horizontal and vertical components of the FoV, $\epsilon_w \times \epsilon_h$, are found from the slit width and height as
\begin{subequations}
\begin{align}
\tan{\left(\frac{\epsilon_{w}}{2}\right)} &= \frac{w_{\text{slit}}}{2F_0}, \label{eq:fov_x} \\
\tan{\left(\frac{\epsilon_{h}}{2}\right)} &= \frac{h_{\text{slit}}}{2F_0}. \label{eq:fov_y}
\end{align} 
\end{subequations}

Assuming no loss of light transmission within the spectrometer from the front to the exit plane, the etendue is expressed as
\begin{equation}
G = \pi \frac{D_0^2}{4 F_0^2}\cos(\beta_c) w_{d} h_{d},
\end{equation}
\noindent where
\begin{subequations}
\begin{align}
h_{d} &= h_{\text{slit}}\frac{F_2}{F_1}, \label{eq:effective_height}\\
w_{d} &= \frac{w_{\text{slit}}F_2}{\cos(\beta_c)F_1} \label{eq:effective_width},
\end{align}
\end{subequations}
\noindent and $\beta_c$ is the diffraction angle at center wavelength $\lambda_c$ \cite{Lerner2006}. It is assumed that $\beta_c\approx\beta(\lambda)$ for all $\lambda$.
% The horizontal and vertical magnification of entrance slit image

The bandpass $BP$ for the optical system, or the recorded Full Width at Half Maximum (FWHM) of a monochromatic spectral
line, and is a measure of the instruments ability to separate adjacent spectral lines in the
spectrogram. Assuming no degradation due to aberrations and diffraction effects, the optical bandpass may be approximated as
\begin{align}
BP &\approx\frac{g w_{\text{slit}}}{\kappa F_1}, \label{eq:bp}
\end{align}
\noindent where $g$ is the grating groove spacing and $\kappa$ is the spectral order \cite{Lerner2006}. 
% For an emission with finite spectral bandwidth such as fluorescent light, in real-life the bandpass is also a function of the natural spectral bandwidth of the emitting source and the limiting resolution of the instrument, where the latter usually has a minuscule effect.
\subsection{Detector}
 $N_x$ lines with $N_y$ and $N_\lambda$ pixels each are collected sequentially during each camera integration time $\Delta t$, to form a threedimensional datacube. $N_y$ is the total number of spatial pixels perpendicular to the scan direction and $N_{\lambda}$ is the total number of pixels along the spectral dimension, cf. Figure \ref{fig:push_scan}. It is assumed that $\Delta t=1/FPS=\tau +\delta t$ is the associated camera integration time where $FPS$ is the frame rate, $\tau$ is the camera exposure time and $\delta t$ is the camera read-out time.
\begin{figure}[htbp]
  \centering
      \includegraphics[width=0.4\textwidth]{figs/fad.png}
  \caption{Detector plane with $h_d$ and $w_d$ being the vertical and horizontal magnification of the entrance slit image. The camera's mechanical layout may block some of the light as shown by the darkest gray regions.}
	\label{fig:fad2}
\end{figure}
The rounded up amount of illuminated pixels in per magnified slit image, as shown in Figure \ref{fig:fad2}, are approximately
\begin{subequations}
\begin{align}
    N_{h} = \frac{h_d}{\Delta p_y}, \\
    N_{w} = \frac{w_d}{\Delta p_\lambda},
\end{align}
\end{subequations}
\noindent where $\Delta p_\lambda$ and $\Delta p_y$ are the width and height of a pixel, respectively.

With the ability to bin pixels, photon-electrons are gathered from adjacent pixels to create one merged pixel with higher SNR at the cost of reduced spectral or spatial resolution. The signal increases proportionally with the square root of number of binning operations $B_{\lambda}$ in the spectral direction or $B_{y}$ in the vertical spatial direction. 
\subsection{Signal-to-Noise Ratio}
The photon flux into the detector may be written as
\begin{equation}
\dot{\Phi}(\lambda) = L(\lambda)\eta_0 \eta_1 \eta_{G}(\lambda) \eta_2 G \lambda\frac{ BP}{h_{\text{planck}}c}, \label{eq:photons}
\end{equation}
\noindent where $L(\lambda)$ is the radiance as a function of wavelength reaching the sensor, $\eta_0, \eta_1, \eta_2$ are the optical efficiencies of the front, collimator and detector lenses respectively, $\eta_G$ is the grating efficiency, $c$ is the speed of light, and $h_{\text{planck}}=6.62607015\times10^{-34}$ $\rm{Js}$ is the Planck constant. 

The number of photons converted to electrons in each pixel is
\begin{equation}
c_{\text{electrons}} = \frac{\eta_{\text{QE}}(\lambda)\dot{\Phi}(\lambda)\tau}{N_{w}N_{h}}, \label{eq:photons2}
\end{equation}
\noindent where $\eta_{\text{QE}}(\lambda)$ is the quantum efficiency of the detector. Assuming that $c_{\text{electrons}}$ has a Poisson probability distribution, then the SNR in one unbinned pixel is
\begin{align}
SNR_{[1, 1]} 
&=\frac{c_{\text{electrons}}}{\sqrt{c_{\text{electrons}} + c_{\text{dark}}+c_{\text{read-out}}^2+c_{\text{dig}}^2}}.  \label{eq:snr}
\end{align}
\noindent where $c_{\text{dark}}=i_{\text{dark}}\Delta t$ is represented with a Poisson probability distribution, while $c_{\text{read-out}}$ and $c_{\text{dig}}$ are assumed to have Gaussian probability distribution with zero mean \cite{Moses2012, Skauli2011}. $i_{\text{dark}}\Delta t$ is the average shot noise registered due to dark current $i_{\text{dark}}$, $c_{\text{read-out}}$ is the standard deviation of electrons due to the sensor read-out circuits, and $c_{\text{dig}}=c_{\text{electrons},\text{max}}/(2^{b}\sqrt{12})$ is the standard deviation of digitization (or quantization) noise where $c_{\text{electrons},\text{max}}$ is the well depth of electrons and $b$ is the Analog-to-Digital Converter (ADC) bit depth.

To match the optical bandpass then we must bin $B_\lambda=\ceil{N_w}$ pixels in the spectral direction where $\ceil{\cdot}$ means rounding up to an integer, rendering the SNR in a $[\ceil{N_w}, 1]$ window to be $SNR_{[\ceil{N_w}, 1]} \approx \sqrt{N_w}SNR_{[1,1]}$. Binning may also be applied for the spatial pixels. In particular, if large distances are covered during each camera exposure time such as for HYPSO-1, the partial overlap of frames in the along-track direction may be utilized to predict the benefits of super-resolution algorithms.

\subsection{Payload Design} \label{sec:payload-hsi}
HYPSO-1's payload, shown in Figure \ref{fig:HSI}, is built with mainly COTS products from Thorlabs and Edmund Optics and a few 3D-printed parts \cite{Sigernes18}. The design provides a spectral range in the visual and near-infrared part of the spectrum and bandpass of $3.33 \hspace{3pt} \rm{nm}$ which fulfills HYPSO-1 mission requirements to observe algal blooms and primary productivity. In theory, the f-numbers should be equal to maximize the light throughput in the optics, however these are set to $F_0/\#=F_1/\#=2.8$ and $F_2/\#=2$ to avoid stray light effects. The instrument's specifications and performance are given in Table \ref{tab:optics}.
\begin{figure}[tbhp]
  \begin{center}
    %\includegraphics[width=70mm,angle=0]{figs/HSI_v6.PNG}
    \includegraphics[width=60mm,angle=0]{figs/HSI.jpg}  %more options in figs/mech folder
    \caption{Hyperspectral imager payload assembled for CubeSat integration}
    \label{fig:HSI}
\end{center}
\end{figure}

The chosen sensor is the SONY IMX249 mounted in an industrial camera head from The Imaging Source Europe GmbH. It has a reported well depth of about 33022 $e^{-}$, equivalent to a maximum SNR of approximately $181.6$ or $45.2 \hspace{3pt} \rm{dB}$ per pixel if not binned.The detector enables a maximum frame rate of up to $FPS=47$, but due to constraints on the data throughput the practical FPS setting is governed by the number of binning operations, subsampling of pixels and the chosen window of pixels, commonly named as the Area of Interest (AoI).
\begin{table}[htbp]
	\caption{Hyperspectral imager specifications}
	\label{tab:optics}
	\centering
			\begin{tabular}{l l}
				\hline
				Parameter & Value \\
				\hline 
				FoV $\epsilon_w \times \epsilon_h$ &	$0.0564^{\circ} \times 7.8826^{\circ}$  \\
				$F_0=F_1=F_2$ & 50 $\hspace{3pt} \rm{mm}$ \\
				$F_0/\#=F_1/\#$ & 2.8  \\
				$F_2/\#$ & 2 \\
				$D_0=D_1$  & 17.9 $\hspace{3pt} \rm{mm}$ \\
				$D_2$ &  25 $\hspace{3pt} \rm{mm}$ \\
				Slit width $w_{\text{slit}}$ & 50 $\hspace{3pt} \mu\rm{m}$ \\
				Slit height $h_{\text{slit}}$ & 7 $\hspace{3pt} \rm{mm}$ \\
				Optical efficiency $\eta_{0}=\eta_{1}=\eta_{2}$ & $0.8$ \\
				Grating efficiency $\eta_{G}$ @$500\hspace{3pt} \rm{nm}$ & 0.73 \\
				Spectral order $\kappa$ & 1 \\
				Groove spacing $g$ & 3333.33 $\hspace{3pt} \rm{nm}$ \\
				Diffraction angle $\beta_c$ & 10.37$^{\circ}$ \\
				Pixel size $\Delta p_\lambda=\Delta p_y$ & $5.86 \hspace{3pt} \mu\rm{m}$ \\
				% Full detector resolution & $1936 \times 1216 \hspace{3pt} \rm{pixels}$ \\
				Usable detector resolution & $1936 \times 1194  \hspace{3pt} \rm{pixels}$ \\
				Quantum efficiency $\eta_{QE}$ @$500\hspace{3pt} \rm{nm}$ & 0.77 \\
				Spectral range &  $271-1006 \hspace{3pt} \rm{nm}$ \\ 
			    Bandpass $BP$ & $3.33 \hspace{3pt} \rm{nm}$ \\
				Dark current $i_{\text{dark}}$ & $0.95 \hspace{3pt} \rm{e}^{-}/\rm{s}$ \\
				Read-out noise $c_{\text{read-out}}$ & $6.93 \hspace{3pt} \rm{e}^{-}$ \\
				Quantization noise $c_{\text{dig}}$ & $2.33 \hspace{3pt} \rm{e}^{-}$ \\
				Max. SNR per pixel (unbinned) & $181.6$ ($45.2 \hspace{3pt} \rm{dB}$) \\
			    ADC bit-depth & $12\hspace{3pt}\rm{bits}$ \\
			 %   AoI & $1280 \times 720$ pixels \\
			 %  Binning $B_\lambda \times B_y$ & $6 \times 1$ \\
			 %  Subsampling & 2 \\
  		    % FPS $\zeta$ & Up to $47$ \\
				\hline
				\end{tabular}
\end{table}
% \begin{figure}[H]
%   \centering
%       \includegraphics[width=0.45\textwidth]{figs/optical_eff.png}
%   \caption{Efficiencies for the grating and sensor SONY IMX249 across the spectral range.}
% 	\label{fig:optical_eff}
% \end{figure}

\section{Sampling} \label{sec:sampling}
\subsection{Geometry}
Shown in Figure \ref{fig:conops}, 
% the origin of the body frame is with the coordinate system defined by triad vectors $\hat{\mathbf{y}}_b$, $\hat{\mathbf{x}}_b$, and $\hat{\mathbf{z}}_b$ that in respective order point along the largest to the smallest principal inertia axes. 
the triad axes $\hat{\mathbf{x}}_b$, $\hat{\mathbf{y}}_b$ and $\hat{\mathbf{z}}_b$ represent the body frame that rotates in orbit. The refraction axis of the hyperspectral imager  is mounted along the $\hat{\mathbf{z}}_b$ axis, the slit height $h_\text{slit}$ is mounted along the $\hat{\mathbf{y}}_b$ axis and the slit width $w_\text{slit}$ is mounted along the $\hat{\mathbf{x}}_b$ axis. The satellite's orbit frame where $\hat{\mathbf{x}}_o$ points along the velocity vector (along-track), $\hat{\mathbf{y}}_o$ points towards the negative orbit normal vector (cross-track), $\hat{\mathbf{z}}_o$ represents the nadir vector which is aligned with the position vector defined in the Earth-Centered-Inertial (ECI) frame. 
% The 2-D reference frame of a pixel on ground is represented with $\Delta x$ and $\Delta y$, being in-track and cross-track resolution respectively.
The rotation of the satellite body relative to the orbit may be represented by the Euler angles $\phi$, $\theta$ and $\psi$ which are the roll, pitch and yaw angles. In addition, the absolute angle between the Line-of-Sight (LOS) vector $\boldsymbol{\rho}$ and $\hat{\mathbf{z}}_o$, is defined as the viewing angle $\gamma$.
% \begin{align}
% \gamma &=\cos^{-1}\left(\frac{1}{\sqrt{\sec^2(\theta)+\tan^2(\phi)}}\right)
% \end{align}
% 
The angular velocities about the satellite body frame, as measured by on-board gyroscope sensors, are represented by $\omega_x$, $\omega_y$, and $\omega_z$. 

% The transformation is undefined for $\theta = \pm \frac{\pi}{2} \pm q \pi$ where $q$ is an integer.\footnote{In practice, it is assumed that the singularity is avoided by imposing $\vert\theta\vert<\frac{\pi}{2}$ during a slew maneuver about the $\hat{\mathbf{y}}_b$ axis as the concerned remote sensing targets are beneath the orbit track and not at or beyond the Earth's horizon.}
The hyperspectral imager's instantaneous footprint, expressed in horizontal and vertical components, are
% \begin{subequations}
% \begin{align}
% P_{w} &= H\sin\Big(\frac{\epsilon_{w}}{2}\Big)\sec\phi\sec\theta \Bigg(\sec\Big(\theta+\frac{\epsilon_w}{2}\Big)\notag\\&+\sec\Big(\theta-\frac{\epsilon_w}{2}\Big)\Bigg), \\
% P_{h} &= H\sin\Big(\frac{\epsilon_{h}}{2}\Big)\sec\theta\sec\phi\Bigg(\sec\Big(\phi+\frac{\epsilon_h}{2}\Big)\notag\\&+\sec\Big(\phi-\frac{\epsilon_h}{2}\Big)\Bigg),
% \end{align}
% \end{subequations}
\begin{subequations}
\begin{align}
P_{w} &= H\sec\phi\bigg(\tan\Big(\theta+\frac{\epsilon_w}{2}\Big)-\tan\Big(\theta-\frac{\epsilon_w}{2}\Big)\bigg), \\
P_{h} &= H\sec\theta\bigg(\tan\Big(\phi+\frac{\epsilon_h}{2}\Big)-\tan\Big(\phi-\frac{\epsilon_h}{2}\Big)\bigg),
\end{align}
\end{subequations}
\noindent which are transformed to the along-track and cross-track components of a central pixel as
\begin{subequations}
\begin{align}
\delta x &\triangleq \cos(\psi)P_{w}+\sin(\psi)\frac{P_{h}}{N_{y}}, \label{eq:footprint_x}\\
\delta y &\triangleq \cos(\psi)\frac{P_{h}}{N_{y}}-\sin(\psi)P_w. \label{eq:footprint_y}
\end{align}
\end{subequations}
Ground-projected pixels near the edge of the swath are elongated compared to the central pixel. Along with effects from Earth curvature, this distortion is known as the "bowtie effect" which may be corrected in image processing \cite{Richards1999, Sayer2015}. We note that the ground pixel size is relatively small (i.e. on a meter-scale) and the combination of high frame rate and narrow FoV renders the pixel elongation and the Earth curvature as seen per pixel to be practically negligible.
\subsection{Spatial Resolution}
Using Eqs. (\ref{eq:footprint_x}) and (\ref{eq:footprint_y}), then the spatial resolution in a pixel acquired during exposure time $\tau$, as shown conceptually in Figure \ref{fig:push_scan}, are expressed in along-track and cross-track components as
\begin{subequations}
\begin{align}
    \Delta x=\delta x+v_{p,x}\tau, \label{eq:spatial1_x} \\
    \Delta y=\delta y +v_{p,y}\tau, \label{eq:spatial1_y}
\end{align}
\end{subequations}
\noindent where $v_{p,x}$ and $v_{p,y}$ are the along-track and cross-track pixel speed as measured on ground
\begin{subequations}
\begin{align}
    v_{p,x} & \triangleq v_{o}
+\dot{\theta}H-\dot{\psi}H\tan(\phi), \label{eq:rotational_vel1} \\
    v_{p,y} & \triangleq -\dot{\phi}H+\dot{\psi}H\tan(\theta), \label{eq:rotational_vel2}
\end{align}
\end{subequations}
\noindent with $v_o$ being the speed of the satellite as measured on ground. 
% The Earth ground speed $v_g$ results in moving features in an image pixel due to the Earth's rotational rate and decreases with higher geodetic latitude of the target area. The Relative Ground Shift (RGS) during a camera exposure time takes into account the relative motion between moving footprint and ground and may be determined by adding the Earth surface speed in the along-track and cross-track distances covered such that $(v_{p,x}-v_{g,x})\tau $ and $(v_{p,y}-v_{g,y})\tau $, respectively. RGS is indicative of motion blur and determines how much a pixel and a surface feature has shifted on ground during an exposure.
\begin{figure}[htbp]
  \begin{center}
    \includegraphics[width=70mm,angle=0]{figs/SGSD.png}
    \caption{Illustration of how SGSD is defined. $p_{\text{ref}}(t_1)$ and $p_{\text{ref}}(t_0)$ denote the reference pixel at time $t_1$ and $t_0$, respectively.} 
    \label{fig:SGSD}
\end{center}
\end{figure}
% The read-out distance, i.e. the distance covered between closing camera shutter and opening it again to capture the next frame, are defined in along-track and cross-track components as $v_{p,x}\delta t$ and $v_{p,y}\delta t$, respectively. In an ideal scenario for a pushbroom imager scanning uniformly in the along-track direction, the cross-track SGSD is zero, i.e. $\tilde{y}=0 \hspace{3pt} \rm{m}$ during $\Delta t = t_1-t_0$. 
Shown in Figure \ref{fig:SGSD}, the SGSD may therefore be defined as the distance between two sequential reference pixels during a camera integration time $\Delta t$, expressed by  along-track and cross-track components, as
\begin{subequations}
\begin{align}
    \tilde{x} & \triangleq v_{p,x}\Delta t, \label{eq:SGSD1} \\
    \tilde{y} & \triangleq v_{p,y}\Delta t. \label{eq:SGSD2}
\end{align}
\end{subequations}
The SGSD also determines the amount of overlap in the set of frames. If the scan direction is aligned with the velocity vector, and since the satellite has high speed as well as $\delta x$ being significantly larger than $\delta y$, it is preferred to slew about the $\hat{\mathbf{y}}_b$ axis to enable better along-track spatial resolution. From Eqs. (\ref{eq:rotational_vel1}) and (\ref{eq:rotational_vel2}) the required angular velocity of the satellite $\omega_{y}$ may be obtained from setting the desired SGSD and vice versa. Alternatively, $\omega_{y}$ and SGSD can be set if a a fixed target length shall be uniformly scanned. 
% if pitch angles at the start and at the end of image acquisition are equal in magnitude but of oppsite sign.
% The required angular velocities for selected integration times are shown in Figure \ref{fig:gsd_desired}.
% \begin{figure}[htbp]
%   \centering
%       \includegraphics[width=0.48\textwidth]{figs/ang_vel_ref.png}
%   \caption{Reference angular rate $\dot{\theta}=\omega_{y}$ vs. desired along-track SGSD assuming $\omega_x=\omega_z=0$, for different integration times $\Delta t$. Altitude is $H=500 \hspace{3pt} \rm{km}$.}
% 	\label{fig:gsd_desired}
% \end{figure}
% \subsection{Slew Maneuver Strategies}
% \subsubsection{GSD-Driven Slew Maneuver}
% Suppose constant and small $\Delta t>0$, $\omega_{x}=\omega_{z}=0$, and $\psi=0$. Re-arranging Eqs. (\ref{eq:rotational_vel1}), the angular velocity of the satellite may be chosen from desired instantaneous in-track GSD $\tilde{x}_{\text{ref}}$ as
% \begin{align}
% \dot{\theta}_{\text{ref}} &= \frac{1}{H}\Big(v_{s}-v_{g,x}+\frac{\tilde{x}_{\text{ref}}}{\Delta t}\Big). \label{eq:desired_inst_gsd_x}
% \end{align}
\subsection{Imaging Strategy} 
Consider the length $s_{g}$ that shall be observed during the time $\Delta T=t_f-t_0$ and the satellite rotating from start to end pitch angles $\theta(t_0)=\theta_0$ and $\theta(t_f)=\theta_f$. Assuming constant altitude, to uniformly scan the target the final pitch angle may be set to $\theta_f=-\theta_0$ such that $\delta x(t_0)=\delta x (t_f)$. The geometry is shown in Figure \ref{fig:orbit-track}. Furthermore, it is assumed that $\omega_{z} = \omega_{x} = 0$, $\phi=\psi=0$ such that $\omega_y=\dot{\theta}$. 
\begin{figure}[htbp]
  \centering
      \includegraphics[width=0.45\textwidth]{figs/orbit_track.png}
  \caption{Geometry of a satellite slewing across a ground target with objective to acquire images along the distance $s_g$. Altitude is $H=500 \hspace{3pt} \rm{km}$.}
	\label{fig:orbit-track}
\end{figure}
The orbit track can be calculated as
\begin{align}
s_{o}&=s_{g}+H\Bigg(\tan\bigg(\theta_0-\frac{\epsilon_{w}}{2}\bigg)-\tan\bigg(\theta_f-\frac{\epsilon_{w}}{2}\bigg)\Bigg).
\end{align}
The time $\Delta T$ required to perform the slew maneuver, can be calculated as
\begin{align}
\Delta T & = \frac{s_{o}}{v_{o}},
\end{align}
and the angular velocity of the spacecraft may found from
\begin{equation}
\omega_{y}=\dot{\theta} = \frac{\Delta \theta}{\Delta T}.
\end{equation}
Figure \ref{fig:slew_angle} shows required angular velocity $\omega_y$ as a function of $\theta_0=-\theta_f$ for varying length $s_g$ to be observed.
\begin{figure}[htbp]
  \centering
      \includegraphics[width=0.48\textwidth]{figs/ang_vel_track.png}
  \caption{Angular velocity $\omega_{y}$ vs. pitch angles $\theta_0=-\theta_f$ for different $s_g$. Altitude is $H=500 \hspace{3pt} \rm{km}$.}
	\label{fig:slew_angle}
\end{figure}
\subsection{Expected Performance}
\subsubsection{Resolution for Nadir-pointing} \label{sec:spacecraft_nadir} 
\begin{table}[htbp]
	\caption{Simulation parameters}
	\label{tab:camera_params}
	\centering
			\begin{tabular}{l r}
				\hline
                Parameter & Value \\
                \hline
                FPS & 22 \\ 
                Camera Integration time $\Delta t$ & $45.4 \hspace{3pt} \rm{ms}$ \\
                Camera Exposure time $\tau$ & $41.4 \hspace{3pt} \rm{ms}$ \\
                Camera Read-out time $\delta t$ & $4 \hspace{3pt} \rm{ms}$ \\
                Target length $s_g$ & $40.08 \hspace{3pt} \rm{km}$ \\
                Altitude $H$ & $500 \hspace{3pt} \rm{km}$ \\
                Satellite speed $v_o$ & $7.61 \hspace{3pt} \rm{km/s}$ \\
                Roll angle $\phi$ & $0^{\circ}$ \\
                Yaw angle $\psi$ & $0^{\circ}$ \\
				\hline
				\end{tabular}
\end{table}
With hyperspectral imager's scan direction being aligned with the along-track direction while pointing at nadir, i.e. $\theta=0^{\circ}$, its instantaneous pixel resolution is $\delta x =500 \hspace{3pt} \rm{m}$. Using the specifications in Table \ref{tab:optics} and camera settings in Table \ref{tab:camera_params}, the obtained spatial resolution is $\Delta x = 815.6 \hspace{3pt} \rm{m}$, $\Delta y =58.6 \hspace{3pt} \rm{m}$ and a swath width of $P_{h}=40.08 \hspace{3pt} \rm{km}$. The along-track SGSD becomes $\tilde{x} =346 \hspace{3pt} \rm{m}$, meaning that $3$ frames partially overlap. It takes $\Delta T = 5.23 \hspace{3pt} \rm{s}$ to scan a target length of $s_g=s_o=40.08 \hspace{3pt} \rm{km}$.
\subsubsection{Resolution for Slew Maneuver} \label{sec:spacecraft_slew}
Using the same parameters in Tables \ref{tab:optics} and \ref{tab:camera_params} are used for imaging during a slew maneuver, Figures \ref{fig:spatial_time} and \ref{fig:cross_spatial_time} show how spatial resolution varies with different pitch rates in ideal settings where no attitude errors are present. Table \ref{tab:SGSD} shows the corresponding SGSD, reference angular velocity and duration for each starting pitch angle $\theta_0$. For example with $\theta_0=20^{\circ}$ as starting pitch angle, the satellite would have to slew at a reference angular velocity of $\omega_{y}= -0.754^{\circ}/\rm{s}$ for $\Delta T = 53.05 \hspace{3pt} \rm{s}$ to cover the target length $s_g=40.08 \hspace{3pt} \rm{km}$ uniformly. The spatial resolution varies between $\Delta x=609.2 \hspace{3pt} \rm{m}$ at $\theta=20^{\circ}$ to $\Delta x= 542.9 \hspace{3pt} \rm{m}$ at $\theta=0^{\circ}$. Additionally, a SGSD of $\tilde{x} = 47.09 \hspace{3pt} \rm{m}$ is achieved instead of $\tilde{x} = 346 \hspace{3pt} \rm{m}$ for the nadir-pointing case. This means there will be at least $12$ frames that partially overlap in the along-track direction, instead of $3$ for nadir-pointing. 

In reality, due to attitude stabilization inaccuracies and system noise, the spatial resolution and SGSD will vary significantly throughout the image acquisition. With reference to the image resolution requirement discussed in Section \ref{sec:mission-design}, in order to have a sequential pixel-to-pixel distance to be less than $100 \hspace{3pt} \rm{m}$, by using Eqs. \ref{eq:fov_x} and \ref{eq:footprint_x}, the attitude error requirement is
\begin{align}
    \vert \delta \theta \vert &< \arctan\bigg(\frac{\vert100-\tilde{x}\vert}{H\sec(\phi+\delta\phi)}+\tan(\theta) \bigg)-\theta, \label{eq:attitude_error}
\end{align}
\noindent which indicates a precise ADCS is required. Figure \ref{fig:dtheta} shows how the required attitude accuracy varies throughout the slew maneuvers with different SGSD.
% For example, even in the best case when SGSD shall be exactly zero at nadir, i.e. $\tilde{x}=0 \hspace{3pt} \rm{m}$ and $\theta=0^{\circ}$, then better than $\pm 0.011^{\circ}$ attitude accuracy is needed. 
With a desired SGSD of $\tilde{x}=47.09 \hspace{3pt} \rm{m}$ at $\theta=20^{\circ}$ and $\phi=0^{\circ}$, using Eq. (\ref{eq:attitude_error}) and assuming $\sec(\phi+\delta\phi)\approx 1$, the attitude accuracy of  $\vert\delta \theta\vert<\pm 0.00535^{\circ}$ is required. For $\theta=10^{\circ}$ and $\tilde{x}=66.96\hspace{3pt} \rm{m}$ then $\vert\delta \theta\vert\pm 0.003672^{\circ}$ is required while for $\theta=30^{\circ}$ and $\tilde{x}=52.59\hspace{3pt} \rm{m}$ then $\vert\delta \theta\vert<0.004075^{\circ}$ is required.
  


% \begin{figure}[htbp]
%   \centering
%       \includegraphics[width=0.45\textwidth]{figs/swath_time.png}
%   \caption{Swath width for different starting/ending pitch angles $\theta(T_0)=-\theta(T_f)$ and angular velocities $\omega_{y}$. Target track is $s_g=70 \hspace{3pt} \rm{km}$.}
% 	\label{fig:swath_time}
% \end{figure}
\begin{figure}[htbp]
  \centering
      \includegraphics[width=0.48\textwidth]{figs/Delta_x.png}
  \caption{Along-track spatial resolution for different pitch angles $\theta_0=-\theta_f$ and angular velocities $\omega_{y}$.}
	\label{fig:spatial_time}
\end{figure}
\begin{figure}[htbp]
  \centering
      \includegraphics[width=0.48\textwidth]{figs/Delta_y.png}
  \caption{Cross-track spatial resolution for different pitch angles $\theta_0=-\theta_f$ and angular velocities $\omega_{y}$.}
	\label{fig:cross_spatial_time}
\end{figure}
% \begin{figure}[htbp]
%   \centering
%       \includegraphics[width=0.48\textwidth]{figs/GSD_x.png}
%   \caption{Along-track SGSD for different starting/ending pitch angles $\theta_0=-\theta_f$ and angular velocities $\omega_{y}$. Target track is $s_g=42.41 \hspace{3pt} \rm{km}$.\hl{tabulate}}
% 	\label{fig:GSDx}
% \end{figure}
\begin{table}[htbp]
	\caption{Along-track SGSD}
	\label{tab:SGSD}
	\centering
			\begin{tabular}{l l l| l}
				\hline
				$\theta_0$ [$^{\circ}$] & $\omega_y$ [$^{\circ}\rm{/s}$] & $\Delta T$ [$\rm{s}$]&  $\tilde{x}$ [$\rm{m}$] \\
				\hline
				0 & 0 & 5.23 & 346  \\
				10 & -0.704 & 28.41 & 66.96 \\
				20 & -0.754 & 53.05 & 47.09 \\
				30 & -0.740 & 81.09 & 52.59 \\
				\hline
				\end{tabular}
\end{table}
\begin{figure}[htbp]
  \centering
      \includegraphics[width=0.48\textwidth]{figs/dtheta.png}
  \caption{Required pitch accuracy throughout the slew maneuvers for different $\theta_0$.}
	\label{fig:dtheta}
\end{figure}
% \begin{figure}[htbp]
%   \centering
%       \includegraphics[width=0.45\textwidth]{figs/GSD_y.png}
%   \caption{Cross-track RGS for different starting/ending pitch angles $\theta(t_0)=-\theta(t_f)$ and angular velocities $\omega_{y}$. Target track is $s_g=70 \hspace{3pt} \rm{km}$.}
% 	\label{fig:GSDy}
% \end{figure}
% \begin{figure}[htbp]
%   \centering
%       \includegraphics[width=0.45\textwidth]{figs/track_time.png}
%   \caption{In-orbit track distance and observation time vs. starting/ending pitch angles $\theta(T_0)=-\theta(T_f)$ and angular velocities $\omega_{y}$. Target track is $s_g=70 \hspace{3pt} \rm{km}$.}
% 	\label{fig:track_time}
% \end{figure}
% \subsubsection{In-track Slew Maneuver with Perturbations}
% Figures \ref{fig:spatial_time_err} and \ref{fig:GSD_x_err} show how noise in the satellite system causes deviations to the sensor output performance and the nominal reference ground track. Noise is assumed to be Gaussian-distributed with zero mean such that noise due to the spacecraft structure and the actuators (e.g. reaction wheels) is represented with $\delta\boldsymbol{\omega} \sim \mathcal{N}(0, 0.1^{\circ}/\rm{s})$. 
% Table \ref{tab:statistics} shows the statistics for spatial resolution, GSD and pitch angle due to variations in vehicle angular velocity $\boldsymbol{\omega}$. 
% \begin{figure}[htbp]
%   \centering
%       \includegraphics[width=0.45\textwidth]{figs/spatial_time_err.png}
%   \caption{In-track spatial resolution for $\Delta \theta = 40^{\circ}$, $\omega_{y}=-0.7025^{\circ}/\rm{s}$ and $\delta\boldsymbol{\omega} \sim \mathcal{N}(0, 0.1^{\circ}/\rm{s})$.}
% 	\label{fig:spatial_time_err}
% \end{figure}
% \begin{figure}[htbp]
%   \centering
%       \includegraphics[width=0.45\textwidth]{figs/GSD_x_err.png}
%   \caption{In-track GSD for $\Delta \theta = 40^{\circ}$, $\omega_{y}=-0.7025^{\circ}/\rm{s}$ and $\delta\boldsymbol{\omega} \sim \mathcal{N}(0, 0.1^{\circ}/\rm{s})$.}
% 	\label{fig:GSD_x_err}
% \end{figure}
% \begin{table}[htbp]
% 	\caption{Statistics for satellite spatial imaging performance during slew maneuver}
% 	\label{tab:statistics}
% 	\centering
% 			\begin{tabular}{l r r r r}
% 				\hline
% 				Value & Mean & Standard Deviation & Maximum & Minimum \\
% 				$\Delta x$ & $549.19 \hspace{3pt} \rm{m}$ & $23.84 \hspace{3pt} \rm{m}$ & $632.46 \hspace{3pt} \rm{m}$ & $442.43 \hspace{3pt} \rm{m}$ \\
% 				$\Delta y$ & $64.28 \hspace{3pt} \rm{m}$ & $21.62 \hspace{3pt} \rm{m}$ & $141.00 \hspace{3pt} \rm{m}$ & $-9.84 \hspace{3pt} \rm{m}$ \\
% 				$P_{y}$ & $71.64 \hspace{3pt} \rm{km}$ & $1.33 \hspace{3pt} \rm{km}$ & $74.71 \hspace{3pt} \rm{km}$ & $70.17 \hspace{3pt} \rm{km}$ \\
% 				$\delta x$ & $510.48 \hspace{3pt} \rm{m}$ & $9.51 \hspace{3pt} \rm{m}$ & $532.37 \hspace{3pt} \rm{m}$ & $500.00 \hspace{3pt} \rm{m}$ \\
% 				$\tilde{x}$ & $38.71 \hspace{3pt} \rm{m}$ & $21.86 \hspace{3pt} \rm{m}$ & $102.60 \hspace{3pt} \rm{m}$ & $-58.25 \hspace{3pt} \rm{m}$ \\
% 				$\tilde{y}$ & $5.36 \hspace{3pt} \rm{m}$ & $21.61 \hspace{3pt} \rm{m}$ & $82.36 \hspace{3pt} \rm{m}$ & $-69.04 \hspace{3pt} \rm{m}$ \\
% 				\hline
% 				\end{tabular}
% \end{table}
\subsubsection{Target SNR}
\begin{figure}[htbp]
  \centering
      \includegraphics[width=0.45\textwidth]{figs/radiance.png}
  \caption{Water-leaving radiance $L_w$ measured by MOBY267 and estimated at ToA for different viewing angles $\gamma$.}
	\label{fig:signal}
\end{figure}
To simulate typical water conditions to be observed by HYPSO-1's hyperspectral imager and its corresponding estimate of SNR, we have used water-leaving radiance measurements from the Marine Optical BuoY (MOBY) with deployment number $267$ off the coast of Hawaii. The data sets are publicly available, mainly used for vicarious calibration of EO remote sensing data \cite{Clark2002}. The chosen measurements are time-stamped at 21:11:38 GMT on 3 July 2019 and a spline curve is fitted to the calibrated data in the wavelength range of $348.8391-749.7629 \hspace{3pt} \rm{nm}$ to match the resolution of the hyperspectral imager. Figure \ref{fig:signal} shows the point measurements and simulated water-leaving radiance for viewing angles $\gamma$ as seen at Top-of-Atmosphere (ToA), assuming that the water-leaving radiance diminishes due to the water refraction index and the atmospheric transmittance consisting of only the Rayleigh optical thickness \cite{Bucholtz1995}. 
% It is also assumed that photon flux is constant during the short exposure time $\tau$. 
% 12 binning, 160 binned pixels, 1936/12 = 160 pixels, 500/160=3.1 nm per binned pixel.
% \begin{figure}[htbp]
%   \centering
%       \includegraphics[width=0.45\textwidth]{figs/MOBY.png}
%   \caption{Water-leaving radiance data collected by MOBY. Red line shows a curve fit to the data at spectral resolution .}
% 	\label{fig:moby}
% \end{figure}
\begin{figure}[htbp]
  \centering
      \includegraphics[width=0.45\textwidth]{figs/snr.png}
  \caption{SNR of $L_w$ as seen at ToA with selected number of binning operations per pixel.}
	\label{fig:fred_snr2}
\end{figure}

Using Eqs. (\ref{eq:photons}), (\ref{eq:photons2}) and (\ref{eq:snr}), Figure \ref{fig:fred_snr2} shows the estimated SNR in the $400-750\hspace{3pt} \rm{nm}$ spectral range for the hyperspectral imager sensing the radiance $L=L_w$ at ToA with $\gamma=0^{\circ}$ and $\tau=41.4 \hspace{3pt} \rm{ms}$. It is shown how binning  pixels in the spectral direction increases the SNR. With no binning and at $B_\lambda=9$ we have $BP=3.33 \hspace{3pt} \rm{nm}$ while $B_\lambda=18$ and $B_\lambda=26$ result in $BP=6.67 \hspace{3pt} \rm{nm}$ and $BP=10 \hspace{3pt} \rm{nm}$, respectively. 
% Blue curve shows SNR per unbinned pixel. Red curve shows SNR for $B_\lambda=N_w$ binned pixels along the spectral direction. Green curve shows SNR for binned pixels with $B_\lambda=3 \times N_w$ along the spectral direction to achieve $BP=10 \hspace{3pt} \rm{nm}$. Without binning in cross-track spatial dimension $B_y$, the optical spatial resolution of a pixel for this case is $\delta x \times \delta y = 500 \hspace{3pt} \rm{m} \times 57.65 \hspace{3pt} \rm{m}$. Magenta curve shows a square window of pixels $9\times 9$ with bandpass of $BP=3.33 \hspace{3pt} \rm{nm}$, that means $B_y=9$ pixels are binned in the spatial direction. Optical spatial resolution for this cases is $\delta x \times B_y\delta y = 500 \hspace{3pt} \rm{m} \times 518.65 \hspace{3pt} \rm{m}$.
It is worth mentioning that the simulated SNR does not take into account the total radiance at ToA which includes light due to predominantly aerosol scattering and sun reflection \cite{Franz2007}. Atmospheric scattering of the solar radiation into the hyperspectral imager's path will typically be 10 to 20 times larger at $500 \hspace{3pt} \rm{km}$ altitude \cite{Corson2008, Gao2012}. With this assumption, $20\cdot L_w$ or $\sqrt{20}$ times the highest SNR of $36.97$ at $463 \hspace{3pt} \rm{nm}$ would be approximately $\sqrt{20}\cdot36.97=165.33$ which is still below the saturation at SNR of $181.6$ for an unbinned pixel. Further, the effective SNR is expected to increase due to more overlapping frames during a slew maneuver. For an ideal slew maneuver with $\omega_y=0.754^{\circ}\rm{/s}$, rendering $12$ overlapping frames, results in up to $\sqrt{12}$ times higher SNR for an image pixel containing the same scene.
% For instance, when observing the radiance $L_w$ presented in Figure \ref{fig:signal} which gives the photon flux of $20703 \hspace{3pt} \rm{e^{-}/s}$ into each pixel assuming ToA radiance is 10 times larger, then the exposure time setting may be constrained to be $\tau<1.59 \hspace{3pt} \rm{s}$.

% For example, by binning $B_\lambda=3 N_\lambda = 26$ and keeping $B_y=1$ then the bandpass becomes $BP=10 \hspace{3pt} \rm{nm}$ and SNR at $495.5 \hspace{3pt} \rm{nm}$ increases with $73 \%$ compared to the case with $B_\lambda=N_\lambda=8.67$ and $BP=3.34 \hspace{3pt} \rm{nm}$. 
% Figure \ref{fig:fred_snr3} shows the calculated SNR of water-leaving target radiance for a square window of pixels $N_\lambda \times N_y = 26 \times 26$ and $BP=10 \hspace{3pt} \rm{nm}$. Optical spatial resolution for this case is $\delta x \times B_y \delta y = 500 \hspace{3pt} \rm{m} \times 1498.1 \hspace{3pt} \rm{m}$.
% \begin{figure}[htbp]
%   \centering
%       \includegraphics[width=0.45\textwidth]{figs/SNR_Fred2.png}
%   \caption{SNR for binned square window of pixels is shown in blue where $N_\lambda \times N_y = 26 \times 26$ and $BP=10 \hspace{3pt} \rm{nm}$. Red curve shows SNR for a window of pixels with $N_\lambda \times N_y = 26 \times 1$ and $BP=10 \hspace{3pt} \rm{nm}$.}
% 	\label{fig:fred_snr3}
% \end{figure}


% As shown in Fig. \ref{fig:snr_lambda}, if there are $13$ overlapping pixels, then SNR at $495.5 \hspace{3pt} \rm{nm}$ is 30.16 and increases with $261 \%$ compared to the case with unbinned pixels. Furthermore if there are four satellites with $13$ overlapping pixels at same altitude $500\hspace{3pt} \rm{km}$ then the SNR is 176.77 which is an increase of $621 \%$ compared to the case with non-overlapped pixels. SNR is then 1.017 at $619 \hspace{3pt} \rm{nm}$ while for non-overlapped case with unbinned pixels the SNR is only $0.2822$ at the same wavelength. If there are additional four pixels that overlap from four other satellites with same viewing angles, then SNR is $>1$ at $652.4 \hspace{3pt} \rm{nm}$.
% \begin{figure}[htbp]
%   \centering
%       \includegraphics[width=0.45\textwidth]{figs/snr_lambda.png}
%   \caption{SNR in a pixel for a spacecraft HSI sensing the Earth water-leaving radiance $L_w$ at $\gamma=0^{\circ}$. The dashed red curve shows the effective SNR in the case of $13$ overlapping pixels. The dashed black curve shows effective SNR when 4 spacecraft are mapping the same target area with 13 overlapping pixels. Exposure time for all spacecraft are $\tau=25 \hspace{3pt} \rm{ms}$.}
% 	\label{fig:snr_lambda}
% \end{figure}
\section{HYPSO-1} \label{hypso-mission}
% \textcolor{blue}{Contribution (3): "We present the chosen spacecraft bus of HYPSO-1 and justify the feasibility of the concept based on the spacecraft capabilities in terms of its chosen subsystems as well as power, communications and data budget constraints."
% Key points to keep in mind while writing this section are:
% \begin{itemize}
%     \item How will the chosen HYPSO-1 bus (present an overview) enable the CONOPS discussed in section \ref{sec:mission-design}?
%     \item What are the key subsystems that impact the CONOPS discussed in section \ref{sec:mission-design}?
%      \item What are the constraints with using the HSI discussed in section \ref{sec:hsi} (and spacecraft constraints on HSI?)?
%     \item Justify with technical system budgets how the CONOPS is feasible as discussed in section \ref{sec:mission-design}
% \end{itemize}}
\subsection{Spacecraft Bus} \label{sec:spacecraft}
% The mission objectives drive the design of the system, and design decisions must be made at the operational, system, functional, logical and the physical levels. 
% \textcolor{red}{ to provide the capabilities required to satisfy the CONOPS discussed in section \ref{sec:mission-design}. However, the COTS-based design is subject to what is available, and trade-offs must be made in terms of performance and functionality for the system design. .}

% This section will describe the spacecraft and the subsystems selected to support the mission, and discuss the constraints and their impact on the system design and CONOPS.
\begin{figure}[htbp]
  \centering
      \includegraphics[width=0.4\textwidth]{figs/M6P.png}
  \caption{Isometric view of M6P with front panel removed showing the hyperspectral imager in the center, a RGB camera to its left and a star-tracker to its right.}
\end{figure}
Due to cost, schedule, and resources available, the hyperspectral imager, described in Section \ref{sec:hsi}, was chosen to be integrated to a commercially available spacecraft bus. The chosen bus is the Multipurpose 6U Platform (M6P) provided by NanoAvionics. The hyperspectral imager must adapt to existing bus interfaces which in turn affects the mission operations. Continuous iterations based on discussions with end users affected the mission and systems design concurrently, a common process in spacecraft development \cite{Ryschkewitsch2009,boehm1988,smad1999}. 
% The feasibility of CONOPS, described in section \ref{sec:mission-design}, can be determined by analysis of the power and data latency budgets.

Among the important subsystems of M6P are the Flight Computer (FC) for onboard data handling and ADCS functions, a SatLab Global Navigation Satellite System (GNSS) for accurate positioning and on-board time synchronization through a Pulse-per-Second (PPS) signal, Electrical Power System (EPS) for power management, a UHF Radio for basic communications, and a Payload Controller (PC) working as a network interface and router between the payload and the bus. All subsystems communicate over a CubeSat Space Protocol (CSP) and Controller Area Network (CAN) network, where each subsystem is a node with its own CSP address. To fulfill the CONOPS described in Section \ref{sec:mission-design}, the M6P is tailored to be equipped with: 
\begin{itemize}
    \item 16 triple junction solar cells made of Gallium Arsenide that charge six Lithium-Ion batteries with a combined energy capacity of approximately $64.9 \hspace{3pt} \rm{Wh}$, providing enough power for the planned sequences.
    \item A BICE NST-1B star-tracker with FoV of $21^{\circ}$ and measuring accuracy of $8 \hspace{3pt} \rm{arcsec}$ along the $x_b$ and $y_b$ axes and STIM 210 IMU with bias stability of $1.32\times 10^{-6}\rm{deg/s}$ and noise standard deviation of $8\times 10^{-3} \rm{deg/s}$, which are dedicated for the slew maneuver to provide precise attitude accuracy. To ensure sufficient settling time in the sensors' temperature-dependent bias and initialization in the attitude estimator algorithm, the sensors are turned on for at least $4 \hspace{3pt} \rm{min}$ prior imaging. Since the sensors consume a lot of power they are therefore scheduled to immediately turn off when the slew maneuver is complete. For operations without hyperspectral imaging, the M6P's six sun sensors, three magnetometers and three Micro-Electro-Mechanical-System (MEMS) gyroscopes are used instead when provide coarser attitude determination to relieve the power budget. 
    \item Four reaction wheels are used for attitude control and provide up to $3.2 \hspace{3pt} \rm{mNm}$ torque each, three being placed orthogonal to each other and the fourth at a $54.7^{\circ}$ angle with respect to the other three. 
    % Two magnetorquers, that may produce a maximum magnetic dipole of $0.46 \hspace{3pt} \rm{Am^2}$ each, are placed along each body axis and are used for reaction wheel momentum dumping.
    \item A RGB camera (IDS UI-125x, 6mm, f/1.4 Ci Series Fixed lens, with custom housing) that takes images with a footprint of $770 \hspace{3pt} \rm{km} \times 540 \hspace{3pt} \rm{km}$ and spatial resolution of approximately $500 \hspace{3pt} \rm{m}$, whose main purpose is to support and validate the orthorectification of hyperspectral images in the spatial domain \cite{habib2016ortho}.
    \item A $2.4 \hspace{3pt}  \rm{GHz}$ Satlab S-band Transceiver provides an usable data rate of up to $1 \hspace{3pt} \rm{Mbps}$ for downlink of payload data to the ground. 
    \item An Onboard Processing Unit (OPU) which is dedicated to fast image processing of hyperspectral data. It based on a Xilinx PicoZed SoC that consists of two core ARM processors and a Field Programmable Gate Array (FPGA). The FPGA enables reconfiguration of hardware for specific applications after launch. The PicoZed interfaces with the RGB and hyperspectral cameras through a customized breakoutboard. Large amounts of payload data need to be transferred from the OPU to the PC at a usable CAN speed of $0.4 \hspace{3pt} \rm{Mbps}$ before transmitting over the radio. It also hosts two SD-cards that allows storing up to $24 \hspace{3pt} \rm{GB}$ of data. 
\end{itemize}

\subsection{Power Budget}
Given that HYPSO-1 is in a $500 \hspace{3pt} \rm{km}$ SSO with Right Ascension of Ascending Node of $198.42^{\circ}$, the comprehensive results for the power budget are shown in Tables \ref{tab:power-budget}. The orbit is shown in Figure \ref{fig:orbit_hypso}.
\begin{figure}[tbhp]
  \begin{center}
    \includegraphics[width=85mm,angle=0]{figs/orbit_hypso.png}
    \caption{The orbit of HYPSO-1 at 10:14:15 on 19 August 2021. The target area is Lofoten, Norway. Selected ground stations at NTNU, KSAT Svalbard and KSAT Spain are represented by white circles.}
    \label{fig:orbit_hypso}
\end{center}
\end{figure}
\begin{table}[htbp]
	\caption{Power budget}
	\label{tab:power-budget}
	\centering
			\begin{tabular}{l r r r}
				\hline
				Battery Capacity & $64.90 \hspace{3pt} \rm{Wh}$ & & \\ 
				Generated Power & $11.65 \hspace{3pt} \rm{W}$ & & \\ 
			   Efficiency & $84.64 \hspace{3pt} \rm{\%}$ & & \\
			   \hline
			  Subsystem & Power ($\rm{W}$) & DC ($\%$) & Power Used ($\rm{W}$) \\
			  \hline
				Hyperspectral imager & 3.675 & 1.054 & 0.039 \\
				RGB camera & 3.360 & 0.019 & $6.384\times10^{-5}$ \\
				OPU imaging & 3.059 & 1.054 & 0.003 \\
			    OPU image processing & 2.315 & 4.76 & 0.146 \\
			    OPU CAN transfer & 2.132 & 35.33 & 0.753 \\
				ADCS normal & 4.765 & 94.77 & 4.516 \\
				ADCS precise & 7.907 & 5.23 & 0.414 \\
			    S-band radio RX & 4.813 & 10.57 & 0.509 \\
				S-band radio TX+RX & 12.201 & 10.57 & 1.290 \\
				Other & 1.153 & 100 & 1.153 \\
				\hline
				Total ($+10\%$ margin) &  & & 9.706 \\
				Remaining &  & & 0.155 \\
				\hline
				\end{tabular}
\end{table}

The M6P solar arrays are able to generate approximately $11.65 \hspace{3pt} \rm{W}$ during $3532 \hspace{3pt} \rm{s}$ out of the total orbit period of $5677 \hspace{3pt} \rm{s}$ (accounting for $2145 \hspace{3pt} \rm{s}$ time in eclipse). It is assumed that the input and output efficiencies of the batteries are $92 \%$ each. Table \ref{tab:power-budget} shows the physically measured nominal power ratings with $5 \%$ component margin and corresponding duty cycles (DC). The OPU, ADCS and S-band radio power ratings are distinguished into more than one operational mode, while "Other" denotes the collective power consumption by FC, EPS, PC and internal bus communications. In particular, high power consumption is expected during the HSI image acquisition and slew maneuver when the IMU and star-tracker are active, consuming up to $1.5 \hspace{3pt} \rm{W}$ each. Adding a $10 \%$ system margin results in remaining available power of about $0.155 \hspace{3pt} \rm{W}$. Constrained by the power budget, allowed downlink time is $10 \hspace{3pt} \rm{min}$ per orbit which means that potentially $75 \hspace{3pt} \rm{MB}$ of data may be downloaded. The allowed for transferring data through CAN is set to $33.42 \hspace{3pt} \rm{min}$ while the time for OPU on-board image processing can be set at a maximum of $270 \hspace{3pt} \rm{s}$.
% As the mission matures other parts of the processing pipeline can become a a part of the on-board processing and then the reconfigurability of the FPGA is a crucial feature to achieve the desired performance when dealing with gigabytes of hyperspectral data.
% \subsubsection{Communications}
% \hl{Roger \\}
% The satellite bus is equipped with two independent communication systems; one Ultra High Frequency (UHF) transceiver and one S-band transceiver. All subsystems can be reached through both radios thanks to the on-board CubeSat Space Protocol (CSP) network. The UHF radio is practical for preparing a high-level telemetry beacon, and giving access to basic telecommands. The S-band radio is primary used for payload data and for software updates. 

% UHF was chosen for ease of operations, especially during commissioning, and S-band was chosen because of the ground segment availability, both commercial and institutional. There was also a large range of S-band subsystems available at a reasonable cos when the subsystem had to be  selected. This gives the highest download rates with largest ground segment availability. X-band was considered but rejected due to a much higher cost. The payload generates a vast amount of data during high resolution imaging, and the radio will be a bottleneck for data downloading. Compression of acquired data will improve this, and reduce the time and energy needed for downloading. 
% \subsubsection{RGB Camera}
% \hl{Joe, Sivert \\}
% The HSI pushbroom camera will be augmented by an RGB camera (IDS UI-125x, 6mm, f/1.4 Ci Series Fixed lens, with custom housing) that takes regular spatial images. 
% The RGB camera will be able to cover an area on ground of 770 $\times$ 540 $km^2$, with a spatial resolution of just under 500 m. 
% Images from the RGB camera will be used as an aid for and as validation of the orthorectification \cite{habib2016ortho} that will be performed to get spatially coherent HSI images.

% In addition to this, the RGB images can be used for pan-sharpening \cite{loncan2015pansharpening}, a technique that would increase the overall spatial resolution of the HSI cubes by inferring spectral values into the higher sampled spatial domain from the RGB images.

% \subsubsection{On-board Processing Unit}
% \hl{Joe, Sivert \\}
% The Onboard Processing Unit (OPU) is based on a Xilinx PicoZed SoC that consists of an ARM processor and a Field Programmable Gate Array (FPGA). 
% The FPGA is a semiconductor device that enables development of features and functions, as well as reconfiguration of hardware for specific applications after launch.
% The PicoZed interfaces with the RGB and HSI cameras through a customized breakoutboard. 
% The two cameras contain firmware to facilitate in image capture and so they are also part of OPU. The OPU is connected to the other satellite subsystems utilizing the CAN network. 

% The OPU will facilitate on-board processing of the hyperspectral data. 
% At the time for launch, the OPU will enable lossless compression by the CCSDS123v1 algorithm\cite{Fjeldtvedt2018}, reducing the HSI image cube size by a factor of 2.5 or more, and making it possible to run such algorithms as part of the on-board processing.
% The purpose of the on-board processing is to enable the pertinent information to be extracted from the hyperspectral images and downlinked within a few passes.

% As the mission matures other parts of the processing pipeline can become a a part of the on-board processing and then the reconfigurability of the FPGA is a crucial feature to achieve the desired performance when dealing with gigabytes of hyperspectral data.
% \subsubsection{Other subsystems}
% \hl{Evelyn, Roger, Mariusz \\}


% The EPS includes \hl{X} number of batteries with \hl{X} $\rm{Wh}$ capacity and has five modes: Full Mode, Normal Mode, Safe Mode, Critical Mode and Hardware Critical Mode, where the subsystem states can be configured to be on or off. The voltage thresholds on triggering the modes are configurable.

%Suggestion


% \subsection{Ground Station Network (Ground Segment)}
% \hl{Roger, Mariusz, Joe \\}
% TO BE MOVED TO CONOPS
% %Only discuss Ground stations... 
% % \begin{figure}[tbhp]
% %   \begin{center}
% %     \includegraphics[width=80mm,angle=0]{figs/HYPSO_GS.png}
% %     \caption{HYPSO Ground Segment Architecture.}
% %     \label{fig:hypso-GS}
% % \end{center}
% % \end{figure}

% In order to facilitate the agility needed for mission operations, the system must make use of ground station gateways (GWS) at different locations to reduce the communication revisit time. The mission operations center (MOC) will be located at NTNU, co-located with one combined S-band and one UHF station. Other S-band stations are available through the KSAT Lite network, as well as through NanoAvionics and partners. 

% The use of the GWS is coordinated by the mission control software (MCS, NanoAvionics). This software acts as a router between the MOC, GWSs and the satellite. The MCS will route data intended for the satellite through the correct GWS during a pass. Data from the satellite is likewise routed from through the GWS and MCS to the end user. 
% \begin{figure*}
%     \centering
%     \includegraphics[width = 0.9\textwidth]{figs/img_processing/pipeline_figure_concept_paper.pdf}
%     \caption{Illustration of proposed imaging pipelines. Minimal for launch, baseline for the first updates, and complete when reached maturity.}
%     \label{fig:image-processing-pipelines}
% \end{figure*}

\begin{figure*}
    \centering
    \includegraphics[width = 0.8\textwidth]{figs/img_processing/image-processing-pipeline.png}
    \caption{Diagram of the proposed imaging processing pipelines. Lightest gray block represents MOBIP, darker gray blocks represent BOBIP, darkest gray block represents TOBIP and black block represents COBIP. The dashed blocks represent modules planned for advanced image processing pipelines.}
    \label{fig:image-processing-pipelines}
\end{figure*}
\subsection{Image processing architecture}
 Figure \ref{fig:image-processing-pipelines} illustrates a diagram of the HYPSO-1's modular image processing architecture which is implemented on the OPU. 
% The modular design allows the operators to switch between the modules as needed. 
The modules can be arranged flexibly with specific orderings that each generate tailored data products, designated as image processing pipelines.
% Binning and subsampling is incorporated into the imager configuration itself to enable higher FPS with lower data traffic and so the most raw data is reduced before the image processing pipelines begin.
The estimated data size reduction factors for each pipeline are shown in Table \ref{tab:data-reduction}.
\begin{table}[htbp]
	\caption{Estimated Data Reduction of Image Processing Pipelines}
	\label{tab:data-reduction}
	\centering
	\begin{tabular}{l |r}
	\hline
	Pipeline & Factor \\
	\hline
    Minimal On-Board Image Processing (MOBIP) & 3 \\
    Baseline On-Board Image Processing (BOBIP) & 14.8 \\
    Target Detection On-Board Image Processing (TOBIP) & 49.9 \\
    Classification On-Board Image Processing (COBIP) & 93.2 \\
    \hline
\end{tabular}
\end{table}
% The planned on-board processing is divided into multiple different pipelines, illustrated in figure \ref{fig:image-processing-pipelines}, that share a basic structure, i.e. some level of processing has to be performed regardless, while others may only improve the timeliness of desired data products or reduce the expected data volume to be down-linked. In the subsequent section a high-level description of the software used and planned in the on-board processing is provided.
% \subsubsection{Settings \& Pre-processing}
% The HSI camera will collect spectrograms at a rate of 15-40 Hz. 
% The data rate is limited by the GigE that connects the camera to the breakout board, but by reducing the area of interest or doing sub-sampling in the spectral dimension, the rate at which frames are collected is adjusted \cite{varntresk2019assembly}. 
% It is expected that the imaging sensor of the HSI camera will have some distortions as a result of the optics, and that the sensitivity of given pixels will change over time. 
% Pre-processing, the initial stage of the imaging pipeline seeks to accommodate these undesirable artifacts.
\begin{table*}[htbp]
	\caption{Uncompressed Data Products*} %\textcolor{red}{Update this Re. Table VI}}
	\label{tab:data-products}
	\centering
	\begin{tabular}{l | r r r r}
	\hline
	& Bands & Pixel size ($\rm{bits}$) & Signatures ($\rm{MB}$) & Total ($\rm{MB}$) \\
	\hline
	Raw & 1074 & $16$& - & 1716 \\
	Binned & 119 & $16$ & - & 190.1 \\
	Dimensionality reduced & 20 & $16$ & - & 32.0 \\
	Classification (16 classes) & 1 & $4$ & 0.003 & 0.40 \\
	Classification (256 classes) & 1 & $8$ & 0.051 & 0.85 \\
	Target detection (only ACE) & 1 & $16$ & - & 1.60\\
	Target detection (with abundance) & 2 & $16$ & - & 3.20 \\
	Target coordinates (top 100) & n/a & $16$ & - & 0.001 \\
	\hline
	\end{tabular}
	
	\begin{tabular}{c}
		*assuming 1168$\times$684 spatial pixels. %\hl{what about spatial compression - jpeg?}
	\end{tabular}
	\vspace*{-\baselineskip}

\end{table*}
\subsubsection{Minimal on-board image processing}
The minimal on-board image processing (MOBIP) pipeline consists of the CCSDS123v1 lossless compression algorithm which is implemented on the OPU's FPGA but can also run on the CPU \cite{Fjeldtvedt2018, orlandic_parallel_2019}. The data size is reduced by a factor of approximately 3. Without loss of spatial or spectral information, the MOBIP data product can be provided to end users who wish to process the data further.  

\subsubsection{Baseline on-board image processing}
The baseline on-board image processing (BOBIP) pipeline adds two important modules to MOBIP, before source encoding and compression are applied. 
The first is a smile and keystone correction, which adjusts images to account for systematic optical and measurement errors inherent to the imager \cite{Henriksen2019}. 

Second is dimensionality reduction, which allows for control of data size while retaining most of the information of the image by adjusting the number of selected reducted bands. \cite{Vit17}. 
% \hl{removing noise-like and redundant components }. 
The smile and keystone correction is applied before the dimensionality reduction to prevent the latter from intertwining systematic, but reversible artifacts irrevocably with the data. For image processing modules beyond BOBIP, dimensionality reduction will increase the computational speed because there will be fewer bands to process \cite{Bakken2019SPIE}.

% \textcolor{red}{For BOBIP, the On-the-Fly-Processing (OTFP) algorithm cam be used as dimensionality reduction because it allows for accurate reconstruction of the hypserspectral information with a minimal number of bands \cite{Vit17}. 
% The residuals from the dimensionality reduction can also be downlinked occasionally and analysed to provide insight about the components removed from the data.
% Once it is rigorously tested and approved, the BOBIP pipeline will provide the nominal data products. 
% For BOBIP the number of selected bands is expected to be about 20.

\subsubsection{Target detection on-board image processing}
Another way to expand MOBIP or BOBIP is to add a target detection (TD) module before compression, named target detection on-board processing (TOBIP) pipeline. Hyperspectral data is amenable to target detection because of its numerous channels \cite{Manolakis2002, Manolakis2005}. One target signature will result in a probability map, or a heat map, of its occurrence across the scene. Multiple target signatures can be represented as separate bands. 
% By incorporating spectral information about the background scene, TD algorithms can locate sub-pixel spectral signatures.
% The 2D heat maps produced by TD are small in data size and can be immediately used to guide in-situ agents without requiring additional processing on ground. 
The Adaptive Cosine Estimator (ACE) is a target detection algorithm for hyperspectral images that is often used to determine the likelihood of a pixel containing a particular spectral signature. 
Both software and software-hardware co-design versions of the algorithm have been developed for OPU. Computing acceleration provided by the OPU's FPGA results in a speedup factor of about $28$ times relative to software implementations \cite{dijehw19_meco}. Effective use of ACE requires a-priori knowledge about the target spectra to be observed. 
% \textcolor{orange}{A spectrum can either be estimated from data in the lab, which is susceptible to calibration inconsistencies between the lab camera and the satellite camera, or it can be estimated directly from the satellite data, so that the target spectrum and input data traverse similar optical paths \hl{?}}. 
A complementary algorithm, the abundance estimator, can be used to estimate how much of a pixel is contained by a target signature. 
For some applications, such as chlorophyll estimation, the abundance estimator itself will be the most relevant component of the data. 

The maxima in a heat map show the pixel locations where it is most probable that a particular target signature is present. Instead of downlinking the entire heat map it is possible transfer only the pixel coordinates of a number of the maxima. If the coordinates are geo-referenced, the in-situ agents can quickly travel to the location of these maxima to inspect in more detail. For example, if the location of the 100 largest local maxima of the map are downlinked together with their latitudes and longitudes, the total data package will only be about $1 \hspace{3pt} \rm{kB}$, shown in Table \ref{tab:data-products}. % The small size of this data product can be quickly downlinked after image acquisition, and for that reason it will enhance the capability of in-situ agents to track dynamic ocean phenomena.
\subsubsection{Classification on-board image processing}
Alternative to TOBIP is the classification on-board image processing (COBIP) pipeline with a classification algorithm that separates the pixels of a hyperspectral image into different classes \cite{Alcolea_2020}. With fewer than 16 classes, it is possible to represent each pixel with a 4 bit integer whereas up to 256 classes can be represented with 8 bits. Representative spectra for each class will also be downlinked. 

Graph-based clustering, an unsupervised method, will be adapted to be run on the OPU as the initial classification algorithm because of its flexibility and it does not require training data \cite{Wang2017}.
Once a database of hyperspectral images is acquired and labelled, a supervised classification algorithm will be incorporated as an alternative.
Recent experiments have shown that gradient boosting decision trees achieve a good balance of accuracy and computational requirements for on-board processing \cite{Alcolea_2020}.
\subsection{Advanced on-board image processing pipelines}
Additional planned algorithms, some of them depicted in dashed blocks in Figure \ref{fig:image-processing-pipelines}, are still in development and may be uploaded on the OPU while HYPSO-1 is in orbit and implemented in a similar successive mission. 
The planned algorithms include image registration, motion-blur correction, geo-referencing, atmospheric correction, and super-resolution, some of them slated to be run on ground. 
% Several of these kinds of algorithms utilize the RGB camera in addition to the hyperspectral imager, so it must be drawn into the image processing pipelines. 
Depending on the end user and target area, different tailored data products will be desired and more modules can be added to the image processing architecture based on these needs. We discuss a few selected algorithms next.

\subsubsection{Image registration and geo-referencing}
In general, even very simple algorithms for registration and geo-referencing require more computational power than is available on-board the HYPSO-1 \cite{Opsahl_2011}. Here, we refer to image registration as the determination of the relative separation of the individual pixels, or sometimes named orthorectification, and geo-referencing as the process of assigning all the pixels to latitude and longitude coordinates. For example, registration is necessary to enable any spatial-spectral methods in image classification, which tend to be more accurate than those that rely only on spectral information. Similarly, to downlink the target detection local maxima coordinates, it is necessary to determine what the latitude and longitude of the relevant pixels. For on-board operation, a simple ray-tracing method will be adapted, which has been prototyped on the ground for joint registration and geo-referencing, similar to the one described in \cite{Schlapfer2002} but with a flat elevation model for the water surface. 
% However, it is expected that more advanced perturbative techniques may be required, and those techniques will initially be run on the ground. 

\subsubsection{Super-resolution}
Super-resolution algorithms are being adapted to enhance the spatial resolution in the images as described in \cite{Park2003}, and may provide high-accuracy and resolution. The initial prototypes for super-resolution model require a measurement process, e.g. determining the point-spread function, and then use that to infer the image at higher spatial resolution \cite{Garrett2019}. The prototypes are based on methods that come from multi-frame super-resolution because of its similarity with the irregularly-sampled data from the slew maneuver \cite{stark_high-resolution_1989, farsiu_fast_2004}. 
Although these super-resolution algorithms do improve the resolution somewhat, they are susceptible to noise, digitization, compression and inaccuracies in the estimate of the point spread function \cite{Baker2002,Kohler2019}. 
However, some of these measurement-based super-resolution methods have previously been successfully applied to remote sensing data \cite{clarisse2019tracking}. 

Prior-based super-resolution techniques overcome the limitations of measurement-based reconstruction techniques by supplementing the input pixels with additional expectations about image statistics. 
Examples of general prior-based techniques include sparse image representations \cite{Yang2010}, and convolutional neural nets \cite{Kim_2016_CVPR,Anwar2020}. 
Some of these produce plausible images at magnification factors of up to 8 times, and even a factor of 6 times. However, these techniques are specifically designed for hyperspectral data. 

On the other hand, dimensionality reduction-based super-resolution techniques unique to hyperspectral imagery have been developed as well \cite{Akgun2005}. Recently, numerous multispectral-hyperspectral fusion techniques have been proposed to enhance the resolution of the latter \cite{Lanaras_2015_ICCV, Yokoya2017}. For utilizing super-resolution in the HYPSO-1 mission, it is necessary to determine how it can be incorporated into a framework with classification or target detection.

\subsubsection{Upload/reprogram Capability}
Both the software and FPGA configurations are planned to be updated during the operation of the satellite \cite{Gjersund2020}. 
The software update must be stringently tested on the ground, both in terms of timeliness and computing resources, before uploading it which can take several passes to complete. 
The OPU retains \textit{golden image}, a version of the operating system and software known to have worked, that it will revert to in case of an update failure. 

\subsubsection{Ground Processing Pipeline}
Similar and extended image processing pipelines should operate on the ground to (a) adjust, fine-tune and prepare data for end users; (b) assist in in-orbit calibration of the hyperspectral imager; and (c) to test algorithms rigorously before uploading them to the satellite for on-board image processing. 
Some of the components of the pipeline such as geo-referencing, atmospheric correction and super-resolution are also amenable to being applied to the data on ground because they require access to reference libraries and are more computationally expensive than the other modules presented.
% \begin{table}[htbp]
% 	\caption{Data budget}
% 	\label{tab:data-budget}
% 	\centering
% 			\begin{tabular}{l r r r r}
% 				\hline
% 				S-band TX Data Rate & $1 \hspace{3pt} \rm{Mbps}$ & & & \\ 
% 				S-band RX Data Rate & $0.04 \hspace{3pt} \rm{Mbps}$ & & & \\ 
% 			   CAN Buffer Speed & $0.4 \hspace{3pt} \rm{Mbps}$ & & & \\
% 			   \hline
% 			    & Upload & TM & Data A & Data B \\
% 			  \hline
% 		    Data Size (MB) & 0.1 & 0.1 & 100.29 & 11.06 \\
% 			CAN Transfer Time (min) & 0.03  & 0.03 & 33.43 & 3.69 \\
% 			Downlink Time (min) & - & 0.01 & 13.37 & 1.47 \\
% 		    Uplink Time (min) & 0.33 & - & - & -  \\
% 		    Orbits Required & 0.0039 & 0.0021 & 2.58 & 0.08 \\
% 				\hline
% 				\end{tabular}
% \end{table}

% \begin{table}[htbp]
% 	\caption{Data Constraints}
% 	\label{tab:data-constraints}
% 	\centering
% 	\begin{tabular}{l r }
%         \hline
%         Available onboard processing time	& $347.26 \hspace{3pt} \rm{s}$\\			
%         Available onboard transfer time &	$2005.80 \hspace{3pt} \rm{s}$\\			
%         Available downlink time &	$1320 \hspace{3pt} \rm{s}$ \\				
%         Maximum data transferred per orbit &	$100.29 \hspace{3pt} \rm{MB}$ \\
%         Total downlink per orbit (incl. headers) &	$165 \hspace{3pt} \rm{MB}$ \\				
%         SD card storage space available	& $24 \hspace{3pt} \rm{GB}$ \\
%         \hline
%     \end{tabular}
% \end{table}
\begin{table*}[htbp]
	\caption{Summary of selected hyperspectral imager mode performance}
	\label{tab:data-types}
	\centering
	\begin{tabular}{l r r r r r}
        Type &	Mode A & Mode B & Mode C & Mode D & Mode E \\ % OK
        \hline
        ADCS Mode &	Slew ($\theta_0=20^{\circ}$) & Slew ($\theta_0=20^{\circ}$) & Slew ($\theta_0=20^{\circ}$) & Nadir & Nadir \\ % OK
        AoI (pixels) &	$1074\times684$ & $1074\times684$ & $1074\times1194$ & $1074\times1194$ & $1936\times1216$\\ % OK
        Binning, $B_\lambda$ (pixels) &	9 & 18 & 9 & 9 & 1 \\ % OK
        % Sub-sampling (pixels) &	1 &	1 &	1 &	1 & 1\\ % OK
        Spectral range ($\rm{nm}$) & $400-800$ & $400-800$ & $400-800$ & $400-800$ & $276-1006$ \\ % OK 
        Spectral bands & 119 & 59 & 119 & 119 & 220 \\ % OK 
        Bandpass ($\rm{nm}$) & 3.33 & 6.67 & 3.33 & 3.33 & 3.33 \\ % OK
        FPS & 22 & 15 & 12 & 12 & 3 \\ % OK
        Exposure time, $\tau$ ($\rm{ms}$) &	41.45 & 49.10 & 49.10 & 49.10 &	49.10 \\ % OK
        Scan duration ($\rm{s}$) & 53.08 & 53.08 & 57.00 & 9.19 & 1.0 \\ % OK
        Number of frames & 1168 & 797 & 685 & 111 & 3 \\
        SNR of target @$471 \hspace{3pt} \rm{nm}$, nadir & 107.87 & 166.52 & 117.75 & 117.75 & 40.00 \\
        Scan distance, along-track ($\rm{km}$) & 40.08 & 40.08 & 69.97 & 69.97 & 7.60  \\ % OK
        Spatial resolution, along-track, nadir ($\rm{m}$) & 542.9 & 550.9 & 573.1 & 873.8 & 873.8  \\
        Swath width, nadir ($\rm{km}$) & 40.08 & 40.08 & 69.97 & 69.97 & 69.97 \\ % OK
        Spatial resolution, cross-track, nadir ($\rm{m}$) & 58.60	& 58.60 & 58.60 & 58.60 & 58.60 \\
        SGSD, along-track ($\rm{m}$) & 47.09 & 69.07 & 124.08 & 634.39 & 2537.56 \\
        Data size, Raw ($\rm{MB}$) & 190.67 & 65.05 & 195.20 & 31.63 & 14.13 \\
        \hline
        Data size, MOBIP ($\rm{MB}$) & 63.56 & 21.68 & 65.07 & 10.54 & 4.71 \\
        Onboard processing time ($\rm{s}$) & 1.64 & 1.22	& 1.65 & 1.11 & 1.05 \\
        CAN transfer time ($\rm{min}$) & 21.19 & 7.23 & 21.69	& 3.51 & 1.57 \\
        Downlink time ($\rm{min}$) & 8.47 & 2.89 &	8.68 &	1.41 & 0.63 \\
        \hline
        Data size, BOBIP ($\rm{MB}$) & 12.82 & 8.82 & 13.12 & 2.13 & 0.51 \\
        Onboard processing time ($\rm{s}$) & 49.30 & 17.48	& 50.45 & 9.01 & 4.58 \\
        CAN transfer time ($\rm{min}$) & 4.27 & 2.94 & 4.37 & 0.71 & 0.17 \\
        Downlink time ($\rm{min}$) & 1.71 & 1.18 & 1.75 & 0.28 & 0.07 \\
        \hline
        Data size, TOBIP ($\rm{MB}$) & 3.82 & 1.30 & 3.91 &	0.63 & 0.28 \\
        Onboard processing time ($\rm{s}$) & 96.97 & 33.74	& 99.25 & 16.92 & 8.11 \\
        CAN transfer time ($\rm{min}$) & 1.27 & 0.43 & 1.30 & 0.21 & 0.09 \\
        Downlink time ($\rm{min}$) & 0.51 & 0.17 & 0.52 & 0.08 & 0.04 \\
        \hline
        Data size, COBIP ($\rm{MB}$) & 2.05 & 0.70 & 2.09 &	0.34 & 0.15 \\
        Onboard processing time ($\rm{s}$) & 96.97 & 33.74	& 99.25 & 16.92 & 8.11 \\
        CAN transfer time ($\rm{min}$) & 0.68 & 0.23 & 0.70 & 0.11 & 0.05 \\
        Downlink time ($\rm{min}$) & 0.27 & 0.09 & 0.28 & 0.05 & 0.02 \\
        \hline
	\end{tabular}
\end{table*}
% \begin{table*}[htbp]
% 	\caption{Data Types}
% 	\label{tab:data-types}
% 	\centering
% 	\begin{tabular}{l r r r r r}
%         \hline
%         Allowed onboard processing time	& $347.26 \hspace{3pt} \rm{s}$ & & & &\\			
%         Allowed onboard transfer time &	$2005.80 \hspace{3pt} \rm{s}$ & & & &\\			
%         Allowed downlink time &	$1320 \hspace{3pt} \rm{s}$ & & & &\\				
%         Allowed data size to transfer per orbit &	$100.29 \hspace{3pt} \rm{MB}$ & & & &\\
%         Allowed to downlink per orbit &	$165 \hspace{3pt} \rm{MB}$ & & & &\\				
%         Allowed to store (SD card)	& $24 \hspace{3pt} \rm{GB}$ & & & &\\					
%         \hline
%         Type &	Data A &	Data B &	Data C &	Data D &	Data E \\
%         \hline
%         ADCS Mode &	Slew &	Slew &	Nadir &	Slew &	Slew \\
%         AoI (pixels) &	$1200\times720$ &	$1200\times720$ &	$1200\times720$ &	? &	?\\
%         FPS &	20 &	20 &	30 &	20 &	20\\
%         Maximum exposure, $\tau$ ($\rm{ms}$) &	50 &	50 &	33 &	50 &	50\\
%         Binning, $B_\lambda$ (pixels) &	3 &	3 &	3 &	? &	12 \\
%         Sub-sampling (pixels) &	4 &	4 &	4 &	? &	1 \\
%         Spectral Bands &	161 &	20 &	161 &	161 &	161 \\
%         Number of frames &	1139 &	1139 &	276 &	1139 &	1139 \\
%         \hline
%         Level 1 data size ($\rm{MB}$) (1/2) &	87.21 &	19.234 &	25.07 &	? &	? \\
%         Time required to transfer ($\rm{min}$)	& 29.07 &	6.4113 &	8.3567 & & \\		
%         Time required to download ($\rm{min}$)	& 11.628 &	2.5645 &	3.34 & & \\		
%         Number of datacubes per orbit &	1.89 & & & & \\				
%         \hline
%         Level 2 data size ($\rm{MB}$) (1/2) &	43.605 & 	9.617 &	12.535 & & \\		
%         Time required to transfer ($\rm{min}$)	& 14.535 &	3.2057 &	4.1783 & & \\		
%         Time required to download ($\rm{min}$)	& 5.814	& 1.2823 &	1.6713 & & \\		
%         Number of datacubes per orbit & & & & & \\					
%         \hline
%         Level 3 data size ($\rm{MB}$) (1/10) &	4.3605 & 0.962 &	1.2535 & & \\		
%         Time required to transfer ($\rm{min}$) &	1.4535 &	0.3207 &	0.4178 & & \\		
%         Time required to download ($\rm{min}$) &	0.5814 &	0.171 &	0.1671 & & \\		
%         Number of datacubes per orbit &	17 & & & & \\ 
%         \hline
% 	\end{tabular}
% \end{table*}

% \begin{table*}[htbp]
% 	\caption{Latency for Mode A Data}
% 	\label{tab:scenario-2b}
% 	\centering
% 	\begin{tabular}{l l l r}
%         \hline
%         Sequence & Start time & End time & Duration ($\rm{s}$) \\	
%         \hline
%         Orbit 1 & & & \\
%         \hline
%     	HSI scan & 10:11:17.00 & 10:12:10.36 & 53.36 \\
%     	Onboard Processing & 10:12:10.36 & 10:12:12.04 &  1.68 \\
%     	CAN Transfer & 10:12:12.04 & 10:34:58.06 & 1366.02 \\
%         Cruise & 10:34:58.06 & 10:52:18.36 & 1040.30 \\
%         Eclipse & 10:52:18.36 & 11:27:10.34 & 2091.98 \\
%         \hline
%         Orbit 2 & & & \\
%         \hline
%         Exit Eclipse & 11:27:10.30 &	11:39:51.17 & 760.87 \\
%         Downlink to KSAT Svalbard & 11:39:51.17 & 11:43:52.46 & 241.29 \\
%         Downlink to NTNU & 11:43:52.46	& 11:48:57.57 & 305.11 \\
%         \hline
%         Total latency ($\rm{hrs}$) & & & 1.63 \\
%         \hline
%     \end{tabular}
% \end{table*}
\begin{table*}[htbp]
	\caption{Latency for Mode A data}
	\label{tab:scenario-2b}
	\centering
	\begin{tabular}{l | l r |l  r|l r|l r}
	    \hline
         & MOBIP & ($63.56 \hspace{3pt} \rm{MB}$) & BOBIP  & ($12.82 \hspace{3pt} \rm{MB}$) & TOBIP & ($3.82 \hspace{3pt} \rm{MB}$) & COBIP & ($2.05 \hspace{3pt} \rm{MB}$) \\		
        \hline
        Sequence & Start time & Duration ($\rm{s}$) & Start time & Duration ($\rm{s}$) & Start time & Duration ($\rm{s}$) & Start time & Duration ($\rm{s}$) \\	
        \hline
        Orbit 1 & & & & & & & & \\
        \hline
    	HSI scan & 10:14:15.00 & 53.08 & 10:14:15.00 & 53.08 & 10:14:15.00 & 53.08 & 10:14:15.00 & 53.08 \\
    	Onboard Processing & 10:15:08.08 &  1.64 & 10:15:08.08 & 49.30 & 10:15:08.08 & 96.97 & 10:15:08.08 & 96.97  \\
    	CAN Transfer & 10:15:09.14 & 1271.16 & 10:15:57.38 & 256.37 & 10:16:45.05 & 76.32 & 10:16:45.05 & 40.93 \\
        Downlink to NTNU & - & - & - & - & 10:18:01.37 & 30.53 & 10:17:25.98 & 16.37 \\
        Downlink to KSAT Spain & - & - & 10:20:13.75 & 102.55 & - & - & - & - \\
        Cruise & 10:36:20.30 & 1316.47 & - & - &- & - & - & - \\
        Eclipse & 10:58:16.77 & 2051.50 & - & - &-  & - & - & - \\
        \hline
        Orbit 2 & & & & & & & & \\
        \hline
        Exit Eclipse & 11:32:28.27 & 601.35 & - & - & - & - & - & - \\
        Downlink to KSAT Svalbard & 11:42:29.61 & 242.65 & - & - & - & - & - & - \\
        Downlink to NTNU & 11:46:32.26	& 265.82 & - & - & - & - & - & - \\
        \hline
        Total latency ($\rm{min}$) & & 96.73 & & 7.69 & & 4.28 & & 3.46 \\
        \hline
    \end{tabular}
\end{table*}
% \begin{table*}[htbp]
% 	\caption{Latency for Mode A Data - w/o CAN overhead}
% 	\label{tab:scenario-2b}
% 	\centering
% 	\begin{tabular}{l l l r}
%         \hline
%         Sequence & Start time & End time & Duration ($\rm{s}$) \\	
%         \hline
%     	HSI scan & 10:11:17.00 & 10:12:10.36 & 53.36 \\
%     	Onboard Processing & 10:12:10.36 & 10:12:12.04 &  1.68  \\
%         Downlink to NTNU &	10:12:12.04 & 10:17:06.50 &	294.46 \\
%         Downlink to KSAT Spain & 10:17:06.50 & 10:21:18.44 &	251.94 \\
%         \hline
%         Total latency ($\rm{min}$) & & & 10.03 \\
%         \hline
%     \end{tabular}
% \end{table*}
\begin{table*}[htbp]
	\caption{Latency for Mode A data - w/o CAN overhead}
	\label{tab:scenario-2c}
	\centering
	\begin{tabular}{l | l r |l  r|l r |l r}
	    \hline
         & MOBIP & ($63.56 \hspace{3pt} \rm{MB}$) & BOBIP  & ($12.82 \hspace{3pt} \rm{MB}$) & TOBIP & ($3.82 \hspace{3pt} \rm{MB}$) & COBIP & ($2.05 \hspace{3pt} \rm{MB}$) \\	
        \hline
        Sequence & Start time & Duration ($\rm{s}$) & Start time & Duration ($\rm{s}$) & Start time & Duration ($\rm{s}$) & Start time & Duration ($\rm{s}$) \\	
        \hline
    	HSI scan & 10:14:15.00 & 53.08 & 10:14:15.00 & 53.08 & 10:14:15.00 & 53.08 & 10:14:15.00 & 53.08 \\
    	Onboard Processing & 10:15:08.08 &  1.64 & 10:15:08.08 & 49.30 & 10:15:08.08 & 96.97 & 10:15:08.08 & 96.97  \\
        Downlink to NTNU & 10:15:09.14 &	243.42 & 10:15:57.38 & 102.55 & 10:16:45.05 &	30.53 & 10:16:45.05 & 16.37 \\
        Downlink to KSAT Spain & 10:19:12.56 &	265.05 & - &	- & - &	- & - & - \\
        \hline
        Total latency ($\rm{min}$) & & 9.39 & & 3.42 & & 3.01 & & 2.77 \\
        \hline
    \end{tabular}
\end{table*}
% \subsubsection{Ground Station Network}

% \subsubsection{Mission Control Center (TENTATIVE)}
% \hl{Roger, Mariusz,  \\}
% \subsubsection{Ground Image Processing}
% \hl{Joe, Sivert \\}
% \subsubsection{Data Dissemination (TENTATIVE)}
% \hl{Joe, Mariusz, Sivert \\}

% \subsubsection{Image Processing - Preliminary Results}
% \hl{Sivert, Joe \\}
% Super-resolution algorithms may be developed to enhance the spatial resolution \cite{Park2003, Garrett2019} and provide improved detectability of features of interest. 
% \subsubsection{Results with Robotic In-situ Agents (TENTATIVE)}
% \hl{Sivert, Joe \\}
\subsection{Data Latency}
% Reaching negative remaining power assumes that HYPSO-1 operates beyond what is required which may be detrimental for the battery charging capacity due to high depth-of-discharge over repeated orbits and shall be avoided, e.g. processing over longer time or downlinking more data than the time allocated.

% Given the condition where all of the following sequences shall be executed in the CONOPS
% \begin{itemize}
%     \item uplinking necessary updates, e.g. scripts with TC and camera settings;
%     \item scanning a $70 \hspace{3pt} \rm{km} \times 70 \hspace{3pt} \rm{km}$ target area while slewing at $\omega_y=0.7025 \hspace{3pt} \rm{deg/s}$ from $\theta=20 \hspace{3pt} \rm{deg}$ to $\theta=-20 \hspace{3pt} \rm{deg}$;
%     \item Buffering acquired payload data to radio if necessary with CAN data rate of $0.4 \hspace{3pt} \rm{Mbps}$ including overhead and margin; and
%     \item Downlinking on-board compressed data immediately to next available ground station with S-band data rate of $1 \hspace{3pt} \rm{Mbps}$ including overhead, $15 \%$ margin data size and assuming $10 \hspace{3pt} \rm{deg}$ ground station antenna elevation.
% \end{itemize}
% \noindent and considering the constraints with power budget, data rates through radio and CAN as well as available ground station passes, then each orbit allows for the HSI to capture up to 1140 frames during a total image acquisition time for 57 seconds and consecutively downlinking the data to the nearby ground stations after acquisition. To enable this, the camera parameters may be set to
% \begin{itemize}
%     \item 25 FPS
%     \item AoI of $1280 \times 720$ pixels
%     \item Binning of $B_\lambda=6$ and $B_y=1$
%     \item Sub-sampling of 2 adjacent pixels in the spectral direction
% \end{itemize}
% which results in a data size of $0.144 \hspace{3pt} \rm{MB}$ per frame. Further compressed data results in the size of $0.072 \hspace{3pt} \rm{MB}$ per frame, latter representing the total size for Data A and Data B in Table \ref{tab:data-budget}. For best case and considering scheduled light-duty operations during eclipse, Data A download may be completed in the next orbit pass, i.e. after approximately $1 \hspace{3pt} \rm{hr}$ and $34 \hspace{3pt} \rm{min}$. Data B may be downloaded in the first pass in $1 \hspace{3pt} \rm{min}$ and $28 \hspace{3pt} \rm{s}$. 
% Because of the large size of hyperspectral data products, the data budget limits the number of frames that can be captured, transferred on-board, and downlinked, and, further, limits the timeliness with which they can be downlinked.
% For example, without any AoI cropping or post-processing (binning, compression), the raw data of 500 frames from one slew maneuver will be about $2.36 \hspace{3pt} \rm{GB}$ which is $4.71 \hspace{3pt} \rm{MB}$ per frame. Using the S-band downlink data rate, this would require more than $5 \hspace{3pt} \rm{hrs}$ of continuous transmission to downlink which is beyond the power budget capability and latency requirements.

% Short time between acquisition and data distribution to end users, specifically in less than $3 \hspace{3pt} \rm{hrs}$, is a critical success criteria for demonstration of the HYPSO-1 mission. 
Table \ref{tab:data-types} shows the selected hyperspectral imager modes and the corresponding performance and data size reduction for MOBIP, BOBIP, TOBIP and COBIP. The choice of AoI and binning operations decides the allowed FPS setting for each mode. Mode A and B provide higher spatial resolution but narrower FoV for a target area size of approximately $40 \hspace{3pt} \rm{km}$ by $40 \hspace{3pt} \rm{km}$, while Mode C and D provide coarser spatial resolution and wider FoV for a target area size of approximately $70 \hspace{3pt} \rm{km}$ by $70 \hspace{3pt} \rm{km}$. Mode A, B, C and D have $1074$ out of $1936$ pixels in the spectral direction and cover the spectral range of $400-800 \hspace{3pt} \rm{nm}$. Mode E, with full AoI, is used in the commissioning phase and for in-orbit sensor characterization and calibration to be performed regularly. "On-board processing time", "CAN transfer time" and "Downlink time" represent the collective time it for the image processing pipelines, completing the transfer of data between OPU to PC at speed of $0.4 \hspace{3pt} \rm{Mbps}$ and completing the downlink of data to ground through S-band radio at speed of $1 \hspace{3pt} \rm{Mbps}$, respectively. For MOBIP it takes up to $1 \hspace{3pt} \rm{s}$ to compress the data and storage of data takes up to $10 \hspace{3pt} \rm{ms/MB}$. An upper limit of allowed processing time is set at $100 \hspace{3pt} \rm{ms/MB}$ for BOBIP, TOBIP and COBIP. 

The hyperspectral imager is chosen to nominally operate with Mode A because its data product is a good compromise between spatial resolution, spectral resolution, SNR and data size. Table \ref{tab:scenario-2b} indicates the duration from acquiring 1334 hyperspectral images of a target area nearby Lofoten, Norway, to complete the download of a MOBIP, BOBIP, TOBIP and COBIP data products by the satellite operator. Results are obtained from simulations \emph{AGI Systems ToolKit (STK)} for the aforementioned orbit parameters where the date is taken to be 15 August 2021 and elevation angles for ground station antennas are $10 \hspace{3pt} \rm{deg}$. It is assumed that image acquisition, onboard processing, data transfer between OPU and PC through CAN, and data downlink through S-band radio are scheduled to be consecutive. None of the durations of each operational phase violate the power budget in Table \ref{tab:power-budget}. Keeping in mind the $3 \hspace{3pt} \rm{hrs}$ latency limit goal, for MOBIP data product with CAN overhead it would take about $1 \hspace{3pt} \rm{hr}$ and $37 \hspace{3pt} \rm{min}$ to completely download data which leaves about $1 \hspace{3pt} \rm{hr}$ and $23 \hspace{3pt} \rm{min}$ slack to further process on ground before finally presenting important and actionable information to the end users. For BOBIP, TOBIP and COBIP the latency is expected to be only a few minutes. In theory, without the CAN overhead as shown in Table \ref{tab:scenario-2c}, the latency for MOBIP would on the other hand be $9 \hspace{3pt} \rm{min}$ and $24 \hspace{3pt} \rm{s}$ and can be downlinked almost immediately after image acquisition.

% \section{Image Processing}
% Readily compressed and reduced data containing a few relevant spectral signatures across geo-referenced coordinates may efficiently be transmitted to ground enabling faster operational response to investigate target(s) in-situ.
\textcolor{blue}{Contribution(4): "To reduce the large hyperspectral data size to improve latency between space segment and end user as well as to save onboard power and provide high accuracy and resolution in the data to make the concept feasible, a carefully designed image processing pipeline is presented with algorithm elements that are implemented both on ground and on-board HYPSO-1. Key objectives are to perform compression, image registration, geo-referencing, super-resolution, classification and target detection on the hyperspectral data." Key points to keep in mind while writing this section are:
\begin{itemize}
\item Present an overview of the required/planned image processing pipeline.
    \item How does compression (CCSDS123v1 and Dimensionality Reduction) \emph{enable} and improve the CONOPS described in section \ref{sec:mission-design}?
    \item How does the image processing pipeline enable accurate data "geometrically" (image registration) and radiometrically (atmos. correction, smile/keystone, calibration)?
    \item How does the image processing pipeline guarantee $<100$ m resolution, RE: section \ref{sec:mission-design}, in the image pixels given the strategy discussed in section \ref{sec:sampling}?
    \item How does the image processing pipeline enable detection of relatively faint optical signatures, e.g. algal blooms - the mission objective?
    \item How does the image processing pipeline enable interpretable data (geo-referencing, classification)?
    \item How does the image processing pipeline handle atmospheric correction?
    \item How do the data and power budget requirements determine the onboard image processing pipeline (especially compression and DR), will it be feasible and is it implementable?
\end{itemize}}
% The HYPSO satellite will process the collected data on board the satellite before downlinking it. 
% \subsection{Why on-board image processing}

The constraint on the HYPSO-1 mission that mandates an on-board image processing pipeline is its limited data downlink budget and power budget. 

Considering that HYPSO will have a radio link to the ground station for just over 6 minutes on an average pass, the full data cube would take over 100 orbits to downlink, or about 7 days, which is longer than HYPSO's real-time data access requirement permits. 
HYPSO-1 includes on-board image processing to reduce the size of the data to downlink, while retaining as much useful information as possible. The on-board image processing as a whole will thus be oriented towards either reducing the size of the total data set or towards extracting useful information into a smaller data package. 
In reducing the size of the data packages, the image processing pipeline will also enable HYPSO-1 to study the dynamics of an oceanographic phenomena, by permitting the same location to be imaged multiple times in a day. 
\subsection{Desired data products}

The HYPSO satellite will produce data both for further analysis on the ground and for interfacing with in-situ agents. 
The dual purposes of the HYPSO data leads to multiple data format definitions: (A) the standard format consists of a compressed datacube which will be analyzed further on the ground and (B) the operational format which consists of directly actionable data that can be used to inform the actions of in-situ agents or for monitoring algal blooms without the need to manually process the data through an intermediate stage. 
The operational data format is flexible: it can consist of chlorophyll estimates, a classification map, a probability heatmap, or other data formats defined during the mission. 
Then, the operational data format listed in table \ref{tab:data-types} which lists the data size after dimensionality reduction to 20 bands, can be thought of as an upper bound on the size of the operation data.

% https://app.diagrams.net/#G1VghNiGzg5It9wcmylNQLFkDfrnwN8lq1

\begin{figure*}
    \centering
    \includegraphics[width = 0.9\textwidth]{figs/img_processing/pipeline_figure_concept_paper.pdf}
    \caption{Illustration of proposed imaging pipelines. Minimal for launch, baseline for the first updates, and complete when reached maturity.}
    \label{fig:image-processing-pipelines}
\end{figure*}

\subsection{Image processing architecture}

A modular image processing pipeline enables HYPSO to downlink the minimal data product, while also permitting the satellite to downlink highly processed and reduced but relevant data (figure \ref{fig:image-processing-pipelines}).
Modules can be interchanged to create new operational data products.
Moreover, different hardware can be utilized for different components of the pipeline. 
For example, compression algorithms, based on the CCSDS123v1 standard, have been developed to run on both the CPU and the FPGA \cite{Fjeldtvedt2018}. 
While the FPGA implementation is faster, the CPU implementation allows for lossy compression, which can increase the data throughput. 
The modular design thus allows the operators to switch between the modules as needed (and the module processing configuration will be stored in the metadata for a particular image). 
However, several specific orderings of the modules are designated as target pipelines in order to maintain constistency among the data products.
Note that $12\times$ binning is incorporated into the image acquisition itself, so the most raw data is reduced before the image processing pipelines begin.
Three of the target pipelines are discussed below. 

% The planned on-board processing is divided into multiple different pipelines, illustrated in figure \ref{fig:image-processing-pipelines}, that share a basic structure, i.e. some level of processing has to be performed regardless, while others may only improve the timeliness of desired data products or reduce the expected data volume to be down-linked. In the subsequent section a high-level description of the software used and planned in the on-board processing is provided.
% \subsubsection{Settings \& Pre-processing}
% The HSI camera will collect spectrograms at a rate of 15-40 Hz. 
% The data rate is limited by the GigE that connects the camera to the breakout board, but by reducing the area of interest or doing sub-sampling in the spectral dimension, the rate at which frames are collected is adjusted \cite{varntresk2019assembly}. 
% It is expected that the imaging sensor of the HSI camera will have some distortions as a result of the optics, and that the sensitivity of given pixels will change over time. 
% Pre-processing, the initial stage of the imaging pipeline seeks to accommodate these undesirable artifacts.

\begin{table*}[htbp]
	\caption{Uncompressed Data Products*}
	\label{tab:data-products}
	\centering
	\begin{tabular}{l | r r r r}
	\hline
	& bands & pixel size (bits) & signatures (MB) & total (MB) \\
	\hline
	Raw & 1200 & $16$& - &984\hspace{10 pt} \hspace{1 pt} \\
	Binned & 100 & $16$ & - & 82.0\hspace{4 pt} \hspace{1 pt} \\
	Dim. reduced & 20 & $16$ & - & 16.4\hspace{4 pt} \hspace{1 pt} \\
	Classification (16 classes) & 1 & $4$ & 0.003 & 0.41 \hspace{1 pt} \\
	Classification (256 classes) & 1 & $8$ & 0.051 & 0.88 \hspace{1 pt} \\
	Target detection (only ACE) & 1 & $16$ & - & 0.82 \hspace{1 pt} \\
	Target detection (with abundance) & 2 & $16$ & - & 1.64 \hspace{1 pt} \\
	Target coordinates (top 100) & n/a & $16$ & - & 0.001 \\
	\hline
	\end{tabular}
	
	\begin{tabular}{c}
		*assuming 1139$\times$720 spatial pixels. \hl{what about spatial compression - jpeg?}
	\end{tabular}
	\vspace*{-\baselineskip}

\end{table*}

\subsubsection{Minimal on-board image processing}
\hl{Joe, Sivert \\}
The minimal on-board image processing pipeline (MOBIP) configuration of the image processing pipeline will typically reduce the size of the data by a factor of 2.5 or more and demonstrate the capability of on-board image processing. 
It consists of image acquisition, compression, and downlinking of the data. 
The compression is implemented on the FPGA of the OPU \cite{Fjeldtvedt2018, orlandic_parallel_2019}. 
Although it is simple, this pipeline forms the basis for all the others. 

\subsubsection{Baseline on-board image processing pipeline}
\hl{Joe, Sivert \\}
The baseline on-board image processing pipeline (BOBIP) configuration adds two important components to MOBIP, before lossless compression. The first of these is a smile and keystone correction, which adjusts the data to account for systematic measurement errors inherent to the imager \cite{Henriksen2019}. 
The second of these is dimensionality reduction, which allows for control of the size of the data package while retaining most of the information of the image. 
The smile and keystone correction is applied before the dimensionality reduction to prevent the dimensionality reduction from modeling systematic, but reversible artifacts from becoming irrevocably intertwined with the data. 

The On-the-Fly-Processing (OTFP) algorithm can be used as dimensionality reduction in order to summarize the spectral information with minimal loss of useful systematic information while simultaneously improving SNR \cite{Vit17}. 
The size of the data package can be controlled by adjusting the number of OTFP bands which are downlinked. 
Thus smile and keystone correction precedes it to avoid imprinting systematic artifacts into the reduced data. 
The residuals from the dimensionality reduction can be down-linked and analysed to provide insight as to what kind of information is being reduced away \cite{Vit17}.
Once it is tested, the BOBIP pipeline will become the standard data format. 
Moreover, dimensionality reduction will increase the speed of modules placed after it in a pipeline because there will be fewer bands to process \cite{Bakken2019SPIE}. 

\subsubsection{Target detection on-board image processing pipeline}
\hl{Joe, Sivert \\}
Another way to expand MOBIP is to add a target detection (TD) module before compression. 
Hyperspectral data is amenable to target detection because of its numerous imaging channels.
By incorporating spectral information about the background scene, TD algorithms can locate sub-pixel spectral signatures. 
The 2D heat maps produced by TD are small enough to downlink data quickly (table \ref{tab:data-types}), and can be immediately used to guide in-situ agents without requiring additional processing on the ground. \hl{Make explicit in Table IV}

The Adaptive Cosine Estimator (ACE) is a target detection algorithm in hyperspectral data that is often used to determine how likely it is that a pixel contains a particular spectral signature \cite{Manolakis2002, Manolakis2005}.
Both software and software-hardware co-design versions of the algorithm have been developed for OPU. 
Acceleration on the FPGA results in a speedup factor of about 28$\times$ relative to software implementations \cite{dijehw19_meco}. 
Effective use of ACE requires spectral knowledge about the targets to be observed. 
A spectrum can either be estimated from data in the lab, which is susceptible to calibration inconsistencies between the lab camera and the satellite camera, or it can be estimated directly from the satellite data, so that the target spectrum and input data are subject to the same limitations. 
A complementary algorithm, the fraction estimator, complements ACE by determining how much of a target is in a given pixel, which leads to the 2 bands seen in table \ref{tab:data-products}. 

\subsection{Developing advanced on-board image processing pipelines}

Some algorithms are still in development, and depending on the end-user that a given pass is targeting, different data products will be desirable.
As part of the reconfigurability of OPU, these future processing pipelines could become a part of the on-going mission. 
These algorithms include image registration, geo-referencing, atmospheric corrections, super-resolution, and classification. 
Several of these kinds of algorithms utilize the RGB camera in addition to the HSI, so it must be drawn into the image processing pipelines. 

Super-resolution algorithms are being adapted to enhance the spatial resolution of remotely sensed images \cite{Park2003, Garrett2019} and to provide improved detectability of features of interest in turn. \hl{this should be emphasized and written about further - crucial for slew maneuver and the core of this paper}
%The latter may give more accurate classification and target detection with fine spatial resolution and high spectral resolution in the image \cite{Manolakis2002, Manolakis2005, Bakken2019SPIE}.


\subsubsection{Upload/reprogram Capability (TENTATIVE)}
\hl{Joe, Sivert \\}
Both the software and FPGA configurations are planned to be updated during the operation of the satellite \cite{Gjersund2020}. First, the software update must be stringently tested on the ground, both in terms of timelinesss and resource usage. Then it must be uplinked to the satellite, which can take several passes. The payload will retain a \textit{golden image}, a version of the operating system and software known to have worked, that it will revert to in case of an update failure. 

\subsubsection{Ground Processing Pipeline}
\hl{Joe, Sivert \\}
An additional image processing pipeline should operate on the ground to (1) prepare data for end users, (2) assist in calibrating the camera in-flight, and (3) to test algorithms before uploading them to the satellite for on-board image processing. 
Some of the components of the pipeline such as geo-referencing and super-resolution are also amenable to being applied to the data after downlinking because they either require access to refence libraries or are computationally intensive.























% \hl{I'm rewriting everything that comes after this - J}

% Readily compressed and reduced data containing the relevant information across geo-referenced coordinates may efficiently be transmitted to ground enabling faster operational response to investigate target(s) of interest in-situ.

% CCSDS123 compression techniques on Field-Programmable-Gate-Array (FPGA) have proven useful for real-time processing and relatively fast lossless data-reduction of large hyperspectral data \cite{Fjeldtvedt2018}.



% \begin{table*}[]
% \begin{tabular}{lllll}
% \textbf{Method}           & \textbf{Purpose} & \textbf{Pipeline} & \textbf{Description}                                      & \textbf{Compression Factor} \\ \hline
% Radiometric Calibration   & Pre-processing   & Baseline          & Ensure correct radiometric intensities in the spectograms & N/A                         \\
% Smile/Keystone Correction & Pre-processing   & Baseline          & Correct for smile and keystone in the spectrograms        & N/A                         \\
% Atmospheric Correction    & Pre-processing   & Complete          & Correct for atmospheric effect on transmitted spectra     & N/A                         \\
% Geo-Refrencing            & Pre-processing   & Compelete         & Correlate each pixel with a location                      & N/A                         \\
% Dimensionality Reduction  & Pre-processing   & Baseline          & Denoising of image cube, data reduction                   & num\_spectra/num\_loadings  \\
% Super-Resolution          & Analysis         & Complete          & Increase spatial resolution of aquired image cube         & TBD                         \\
% Target Detection          & Analysis         & Complete          & Probability map of target                                 & num\_spectra/num\_targets   \\
% Chlorophyll-a             & Analysis         & Complete          & Estimate the chl-a content in an area                     & num\_spectra                \\
% Classification            & Analysis         & Complete          & Divide the data into spectrally distinct classes          & num\_spectra/num\_classes   \\
% CCSDS123v1                & Compression      & Minimal           & Lossless compression of image cube                        & $\sim$3                     \\
% Dimensionality Reduction  & Compression      & Baseline          & Lossy compression of image cube                           & num\_spectra/num\_loadings  \\
% Source Encoding           & Compression      & Baseline          & Lossless compression of data for transmission             & TBD                        
% \end{tabular}
% \end{table*}







\section{Conclusions} \label{sec:conclusions}
% Ocean color remote sensing is important for understanding the wellbeing of worldwide ecosystems and maritime environment. Spontaneous Harmful Algal Blooms are colorful processes with large spatial extent. These frequently cause detrimental effects on environment and sustainable aquacultural resources thus demanding high-resolution data from selected target areas that are quickly delivered after detection. Observing such phenomena requires low data latency and high spectral, spatial and temporal resolution. 
The HYPSO-1 mission and systems design shows that COTS-built hyperspectral imagers can be implemented in small-satellites for ocean color remote sensing applications, thus decreasing the development time and lowering costs for such missions. If used appropriately and with the capability of on-board image processing, these type of imagers may provide data products with sufficiently high spatial and spectral resolution as well as low latency to end users from for example the ocean color community or aquaculture industry. 
% Pushbroom hyperspectral imaging produces lines of pixels with numerous narrow spectral bands where the spatial resolution in the images may be improved by utilizing a small-satellites' ability to perform a slew maneuver during image acquisition.

Pushbroom hyperspectral imaging on a small-satellite have challenges in obtaining adequate image quality due to the smaller optics but can be amended by utilizing the small-satellite's system capabilities, in particular by rotating the camera's footprint by performing a smooth slew maneuver to improve the spatial resolution and increase the effective Signal-to-Noise Ratio (SNR) as more overlapping frames are obtained. The figures of merit presented in this paper such as optics size, spatial resolution, spectral resolution, swath width, SNR, Sequential Ground Sampling Distance, data latency as well as spacecraft angular velocity and attitude accuracy, can be used for systems trade-off studies in preliminary systems design of a spacecraft mission. This ultimately enables better efficiency in mission operations and higher performance of small space-based camera systems. 

Tailored image processing pipelines running on a FPGA on-board HYPSO-1, that include CCSDS123v1 lossless compression, dimensionality reduction, target detection, and classification, may reduce the data size considerably without losing important information and resolution. This enables quick download of the data to satisfy any immediate need of the end user, while relieving the power budget. Data products shall be validated by in-situ measurements from autonomous aerial, surface and underwater vehicles and may also be used to guide these to interesting locations. Advanced image processing algorithms under development, such as image registration, geo-referencing, atmospheric correction, super-resolution and chlorophyll estimation, shall be uploaded to the HYPSO-1's reprogrammable FPGA once in orbit. Based on lessons learned from the HYPSO-1 mission, the image processing pipelines will have enhanced and extended capabilities along with better design iterations on the hyperspectral imager for a prospective HYPSO-2 mission and a constellation of dedicated hyperspectral imaging satellites. 
\section*{Acknowledgments}
This work was supported by the Research Council of Norway, Statoil, DNV GL and Sintef through the Centers of Excellence funding scheme, Grant 223254 - Center for Autonomous Marine Operations and Systems (AMOS) and the Research Council of Norway through the IKTPLUSS programme grant 270959 (MASSIVE). This work is also supported by the Norwegian Space Centre contract SAT.01.17.7. JLG acknowledges funding from the European Research Council on Informatics and Mathematics (ERCIM) postdoctoral fellowship.

The authors would like to thank Raphe Kudela at the Ocean Sciences Department, University of California Santa Cruz, Rick Stumpf at National Oceanic and Atmospheric Administration (NOAA) and Ajit Subramaniam at the Lamont-Doherty Observatory, Columbia University for their guidance and valuable discussions on the concept. Authors would also like to thank Geir Johnsen at Department of Biology, NTNU, for his review on the remote sensing requirements, Torbjørn Skauli at Norwegian Defense Research Establishment (FFI) for his review on  practical spectroscopy, Fernando Aguado Agelet at the Department of Telecommunications Engineering, University of Vigo, and Cecilia Haskins at Department of Mechanical Engineering, NTNU, for their input on systems engineering in the project, Annette Stahl and Dennis Langer at Department of Engineering Cybernetics, NTNU, for their input on image processing, Gara Quintana for her help on radio communications and developing link budgets, and Harald Martens and Petter Rossvoll at IdleTechs for their recommendations on hyperspectral data size reduction. 
%Authors are also  grateful for the HSI mechanical design work by Tord, Tuan and Henrik as part of their Master theses at Department of Mechanical Engineering, NTNU.

\bibliographystyle{IEEEtran}
\bibliography{Reference}

\end{document}