 \section{Introduction} \label{sec:intro}
% As a sink for green-house gases and as the environment for marine life and resources, the oceans' role and evolving state, is undeniable. 
% The influence of the changing climate and its impact on the Earth, where 70\% is covered by water, needs to be studied from several perspectives. This ranges from the fine scale, i.e. micro-biology, to the larger scale, like atmospheric phenomenon such as hurricanes, the extent of the global ice melt, and algal blooms. A variety of these phenomena can be detected from space by optical or radar measurements. 
Optical remote sensing is normally used in the context of observing colorful processes with large spatio-temporal extent such as algal blooms. A primary light-absorbing substance in the oceans is chlorophyll, which is involved in phytoplankton photosynthesis and provides clear water surface signatures \cite{Geir2011}. Other substances particularly those composed of colored dissolved organic matter (CDOM) and suspended matter (SM) mix in the turbid waters and also have distinguishable optical characteristics. In particular, colorful algal blooms have a significant impact on the coastal environment, marine life and the society. These sporadically appear worldwide \cite{jessup09} with their size ranging from tens to hundreds of square kilometers and have varying biomass concentrations. Harmful Algal Blooms (HABs) may cause great damage to sustainable human food sources, especially the seafood industry such as fisheries and aquaculture. Given an early warning, the damage can be mitigated if the blooms are detected in due time \emph{before} reaching the fish pens. 
% (for example, operators may increase the amount of oxygen at the bottom of the fish pens or physically moving them away). 

Hyperspectral imaging is a promising method for detecting and monitoring algal blooms, and can create informative data with hundreds of narrow spectral bands \cite{Kutser2006}. Phytoplankton coloration is highly variable and often categorized as “red tides”, “green tides” or “brown tides” in the wavelength range of $400-700 \hspace{3pt} \rm{nm}$ \cite{Kutser2006, Johnsen1997, Geir2011, IOCCG2014}. Since the absorption spectra of substances are characteristic, hyperspectral imaging makes it possible to identify the primary production of algal blooms and also the overall health of the ocean based on its physical and chemical composition. According to \cite{IOCCG2014}, plankton and algae types or even species may be directly distinguished or inferred by the correlations with geo-physical parameters. For example, hyperspectral data may reveal the subtle spectral inflections imparted by specific pigment complements. With high spectral resolution, the volume of obtainable information may be increased by a four-figure factor as compared to multi-spectral data products \cite{Ortenberg2011}. Important information about aerosols and water vapor in the sensor's optical path through the atmosphere may be retrieved as well. The highly informational and operational characteristics would enhance the solution level of space monitoring tasks \cite{Villafranca2012}. However, keeping in mind the constraints imposed by the physics of optics on the signal-to-noise ratio (SNR), it is also challenging to downlink the large amounts of data generated by such missions \cite{guelman2009small}. 

Earth observation (EO) satellites operated by agencies such as The National Aeronautics and Space Administration (NASA)  and The European Space Agency (ESA) provide excellent data that cover the Earth on a global scale \cite{knight2014,Aguirre2007}. These have medium to high spatial resolution, but have low spectral resolution and several days before revisiting selected targets \cite{Ack16}. Furthermore, algal blooms may develop or disappear faster than the revisit time of a single satellite, and the data requires careful analysis, processing and validation which usually results in a slow delivery of data products to the end users and public. For example, Sentinel-3’s tailored multispectral sensor and temporal resolution as a stand-alone system is insufficient for monitoring algal blooms without using traditional and often time-consuming methods to retrieve important information \cite{Ogashawara2019}. On the other hand, hyperspectral remote sensing missions are many \cite{Gue16, Sou16, dierssen2015space, Guanter2015, 2014RemS66790M, Keith14, guelman2009small, pearlman2003}, and large satellite EO missions such as the Plankton, Aerosol, Cloud, Ocean Ecosystem (PACE) mission show great promise for utilizing hyperspectral imaging in ocean color applications in general \cite{Werdell2019}.  A small-satellite, often categorized as nano- and micro-satellite \cite{Gue16}, have a shorter lifetime than a large satellite, but can frequently be replaced with updated technology and have lower development and production costs \cite{modern_small_satellites}. 

Single-purpose small-satellites that are suitable for hyperspectral imaging may provide high spatial resolution by observing dedicated target areas of interest. Choosing to observe smaller target areas enables dedicated overlay of each satellite's image. Even though small cameras are limited in spatial resolution when used in Low-Earth-Orbit (LEO), the combination of precise Attitude Determination \& Control System (ADCS) and high camera frame rate may increase the spatial resolution in the images by ensuring sufficient number of overlapping frames throughout the observation. Small-satellites effectively serve as complementary platforms to Autonomous Underwater Vehicles (AUVs), Unmanned Surface Vehicles (USVs), Unmanned Aerial Vehicles (UAVs) and buoys which are limited in mobility and speed \cite{Dic05}. 

With the need for more data containing high resolution and accuracy from specific coastal regions, we present a concept of operations for an upcoming small-satellite mission developed at The Norwegian University of Science and Technology (NTNU), named HYPSO-1 which is a 6U CubeSat. The contributions in this paper are (a) the design of a COTS-assembled pushbroom hyperspectral imager used in the HYPSO-1 mission; (b) the concept of HYPSO-1's remote sensing strategy which enhances the spatial resolution and SNR in the hyperspectral data by performing a steady slew maneuver throughout image acquisition; and (c) We present HYPSO-1's on-board image processing pipeline that mainly aims to reduce data latency between image acquisition and end user, provide high-resolution tailored data products, and consequently alleviate the satellite's power budget. 

This paper is organized as follows. Section \ref{sec:mission-design} describes the ocean color requirements that motivate the choice of imager technology, the HYPSO-1 Concept of Operations (CONOPS) and some key camera and spacecraft capabilities. Section \ref{sec:payload-hsi} presents the design, physics and performance of the chosen pushbroom hyperspectral imager. Section \ref{sec:sampling} describes the remote sensing strategy of how the hyperspectral imager shall be used in practice along with the expected results from numerical simulations. In Section \ref{hypso-mission}, we present HYPSO-1 spacecraft bus and image processing pipeline and justifying the feasibility of the concept based on its subsystems, power budget and analysis on data latency for different chosen camera modes and image processing levels. Finally, we conclude our findings in Section \ref{sec:conclusions}.