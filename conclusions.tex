\section{Conclusions} \label{sec:conclusions}
% Ocean color remote sensing is important for understanding the wellbeing of worldwide ecosystems and maritime environment. Spontaneous Harmful Algal Blooms are colorful processes with large spatial extent. These frequently cause detrimental effects on environment and sustainable aquacultural resources thus demanding high-resolution data from selected target areas that are quickly delivered after detection. Observing such phenomena requires low data latency and high spectral, spatial and temporal resolution. 
The HYPSO-1 mission and systems design shows that COTS-built hyperspectral imagers can be implemented in small-satellites for ocean color remote sensing applications, thus decreasing the development time and lowering costs for such missions. If used appropriately and with the capability of on-board image processing, these type of imagers may provide data products with sufficiently high spatial and spectral resolution as well as low latency to end users from for example the ocean color community or aquaculture industry. 
% Pushbroom hyperspectral imaging produces lines of pixels with numerous narrow spectral bands where the spatial resolution in the images may be improved by utilizing a small-satellites' ability to perform a slew maneuver during image acquisition.

Pushbroom hyperspectral imaging on a small-satellite have challenges in obtaining adequate image quality due to the smaller optics but can be amended by utilizing the small-satellite's system capabilities, in particular by rotating the camera's footprint by performing a smooth slew maneuver to improve the spatial resolution and increase the effective Signal-to-Noise Ratio (SNR) as more overlapping frames are obtained. The figures of merit presented in this paper such as optics size, spatial resolution, spectral resolution, swath width, SNR, Sequential Ground Sampling Distance, data latency as well as spacecraft angular velocity and attitude accuracy, can be used for systems trade-off studies in preliminary systems design of a spacecraft mission. This ultimately enables better efficiency in mission operations and higher performance of small space-based camera systems. 

Tailored image processing pipelines running on a FPGA on-board HYPSO-1, that include CCSDS123v1 lossless compression, dimensionality reduction, target detection, and classification, may reduce the data size considerably without losing important information and resolution. This enables quick download of the data to satisfy any immediate need of the end user, while relieving the power budget. Data products shall be validated by in-situ measurements from autonomous aerial, surface and underwater vehicles and may also be used to guide these to interesting locations. Advanced image processing algorithms under development, such as image registration, geo-referencing, atmospheric correction, super-resolution and chlorophyll estimation, shall be uploaded to the HYPSO-1's reprogrammable FPGA once in orbit. Based on lessons learned from the HYPSO-1 mission, the image processing pipelines will have enhanced and extended capabilities along with better design iterations on the hyperspectral imager for a prospective HYPSO-2 mission and a constellation of dedicated hyperspectral imaging satellites. 