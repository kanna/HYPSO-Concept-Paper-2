\section{S/C Bus} \label{sec:bus}
\hypso will be of a 6U standard CubeSat configuration and will include the following subsystems:
\begin{itemize}
\item Solar panels
\item Electrical Power System (EPS)
\item Remove-Before-Flight (RBF) System
\item Flight Computer (FC) and Attitude Determination \& Control System (ADCS)
\begin{itemize}
\item Magnetorquers (3-axis)
\item 4$\times$ reaction wheels
\item 6$\times$ sun-sensors
\item 3$\times$ gyroscopes
\item IMU
\item 3$\times$ magnetometers
\item Star-tracker
\end{itemize}
\item GNSS/GPS
\item S-band Radio \& patch antenna
\item UHF Radio, turnstile antenna \& monopole antenna
\item Batteries ($>$94 Wh)
\end{itemize}
Secondary payloads are
\begin{itemize}
\item Software-Defined Radio (SDR)
\item RGB Camera
\end{itemize}
Interfacing is prospected to be standard CAN, I$^2$C, SPI and Serial between the subsystems. Figure \ref{fig:mass} shows the estimated mass budget for a 6U CubeSat with the required capabilities.
\begin{figure}[htbp]
  \begin{center}
    \includegraphics[width=80mm,angle=0]{figs/mass_budget.png}
    \caption{Mass Budget for HYPSO}
    \label{fig:mass}
  \end{center}
\end{figure}
%\begin{table}[htbp]
	%\caption{Mass Budget for a 6U in grams}
	%\label{tab:mass}
	%\centering
		%\begin{tabular}{|l|r|r|}
			%\hline
			%Subsystem & Mass, g & Mass, g (+20 \%)	\\ 
			%\hline 															
		%HSI &	800 &	960 \\
		%Structure	& 1060 &	1272 \\
		%Stack rods &	123 &	147.6 \\
		%Mechanisms & 60 &	72 \\
		%Wires \& cables &	200 &	240 \\
		%Reaction Wheels &	940 &	1128 \\
		%Star-tracker &	560 &	672 \\
		%Fine Sun Sensor	& 18 &	21.6 \\
		%Magnetorquers	& 156 &	187.2 \\
		%ADCS &	112	& 134.4 \\
		%Antenna S-band &	110 &	132 \\
		%Antenna UHF/VHF	& 90 &	108 \\
		%Radio S-band	& 65.3 &	78.36 \\
		%Radio UHF/VHF	& 24.5 &	29.4 \\
		%GPS	& 24 &	28.8 \\
		%OBC	& 94	& 112.8 \\
		%EPS	& 100	& 120 \\
		%Batteries	& 500 &	600 \\
		%Solar Panels	& 758	& 909.6 \\
		%SDR	& 209	& 250.8 \\
		%\hline
		%Total	& 6003.8 &	7204.56 \\
		%\hline
		%Total (+20\%)	&  &	8645.47 \\
		%\hline 
		%\end{tabular}
%\end{table}