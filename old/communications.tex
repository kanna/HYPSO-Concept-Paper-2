\section{Communications}

%\hl{Joao, Milica, Mariusz and Roger: we should make some calculations
%  and simulations to get more accurate estimates on minimum and desired
%  data rates}.

In order to save time during the production/integration phase, it is advised that the payload downlink (and other communication systems) will be procured as a part of the satellite bus. Various suppliers will have different equipment and possibly also different requirements/options/offerings for ground stations. This goes both for the comm systems (TM/TC, payload downlink, ground stations) as well as software for operation. 

\subsection{Space segment}
As shown, a S-band payload (2 GHz-band) downlink that is capable of 1-2 Mbps will be able to downlink even uncompressed data sets in less than six minutes. Most CubeSat suppliers have such products available. Downlinking at 1-2 Mbps implies pointing of the satellite antenna, at a lower rate (100 kbps) communication should be possible regardless of spacecraft pointing. X-band systems also exist, but are less available. Therefore, and as S-band seems to have enough capacity, X-band is not considered further. 

S-band systems also have an output power that can be supported by the CubeSat. In practice, this output power will be quite low (typical 1-10 W). In order to close the link budget, such systems will require an antenna dish of some size (TBC), perhaps with a diameter of 2 to 4 meters mounted on a steerable mast. This must be verified with a link budget calculation when the specifications of the space segment sub-system hardware is known. 

During planning and operations, the amount of data generated by the payload must controlled in order to meet the capabilities of the downlink. The communication system might be a bottleneck, so scheduling and operation planning must be done in an efficient manner in order to maximize the amount of data that can be downloaded. 

\subsection{Ground Segment}
Several architecture options are possible; a) either only own ground station at NTNU (possibly enhanced with ground stations at partner universities and organizations), b) use of commercial stations or c) a combination of a)\&b). 

\paragraph{Local university ground stations}

The ground station hardware/software could, depending on cost/schedule/options be based on GS hardware/software from the satellite supplier or be composed by general equipment (or both). The ground station will consist of antennas (dish for S-band, Yagi for VHF/UHF), amplifiers and switches as well as a radio supported by operation software. 

\begin{figure}[htbp]
	\centering
		\includegraphics[width=0.50\textwidth]{figs/gs_schem.png}
	\caption{Schematic of multiband VHF/UHF-station at NTNU}
	\label{fig:gs_schem}
\end{figure}


A general schematic of a multi-band ground station is shown in Figure \ref{fig:gs_schem} (S-band not currently included). Several academic/open-source designs exists. The design proposed by master student {\O}yvind Karlsen \cite{karlsen_2017, ark_2017} can be used as a baseline for further study. 


Another installation of a versatile ground station based on the use of SDR and GNURadio are being set up at by the university radio club in Trondheim.
Partners such as Porto, Vigo and other universities can be potential supporters and help with data acquisition. However, cost and time to build ground stations must be considered. 

%Access to NASAs large antennas could be very interesting, especially if things does not work out as it should. This can also be of interest wrt. setting up more advanced experiments based on the SDR-payload in next steps.
Setting up own ground stations also has a value by increasing the competence of the organization. 

University ground stations should also seek to participate in global ground station network initiatives, such as ESAs revived GENSO-network \footnote{http://www.esa.int/Education/Global\_Educational\_Network\_for\_Satellite\_Operations} or the SATNogs-project \footnote{https://satnogs.org/}. The NTNU ground station currently implements and participates in SATNogs.

%In addition, these stations can be used by other missions and might also be included in global ground station networks such as SATNogs or the ESA rebooted GENSO-initiative when they are not used by the project. 

\paragraph{Commercial stations}
The use of for example the KSAT Lite should be considered. This might lead to a cost during operations. However, it can be wise to have a backup communication architecture available if the local ground stations cannot be used for some reason.  

%\redtext{Should probably not go public with this before a meeting}
%\blue
%Another option to explore is to partner with Statsat and gain access to Kystverkets station in Vardø. This will also have a cost for the operation. 
%Both KSAT and Statsat are Norwegian companies with stations far north. This is highly advantageous for the operations and might be preferred for this reason. 

\black
There are also other companies such as Leaf Space \footnote{https://leaf.space/leaf-line/} providing ground station services for small satellite missions. 
Cost and possibilities for all options must be clarified in the next project phase. 

\paragraph{Combination}
In order to quickly downlink data after an image acquisition, multiple ground stations might be desirable. If there is a lack of university partners, the need for a commercial partner can be strong in order to get as much contact time to the satellite as possible. Also, if the satellite during image acquisition moves South to North, a ground station well north of the target should available (e.g. Svalbard). 

Another factor to be considered, is the minimum elevation angle needed to close the link budget. Trondheim is far north and will most likely see up to 5 to 6 passes every day (for UHF TM/TC), but S-band downlink might require higher elevation angles in order to close the link budget (TBC) so only 1-3 passes pr. day might be usable. 

Stations at Svalbard and Vard{\o} are far north and will therefore see the satellite more times during the day, this alone can be a reason for having to use these stations during mission acquisition operations. Local ground stations can be used for TM/TC and simple operations in order to not incur more costs than necessary. 

Further accurate link budget calculations should be performed when more of the system components are known. 

%\subsection{Coverage and access}
%
%From a 500 km, 10:00am SSO orbit, we can derive the following access times\footnote{Based on simulations in STK.}. 
%
%With 10 deg min elevation (passes per day): \\
%Svalbard: 3 – 8 min (11), Trondheim: 2 – 8 min (8), Porto: 3 – 8 min (4)
%With 30 deg min elevation, same period: \\
%Svalbard: 1 – 4 min (6), Trondheim: 2 – 4 min (4), Porto: 2 – 4 min (2)

\subsection{Ground station and operations software}

The satellite bus suppliers will, depending on our needs, deliver software for operating the satellites. When their specification is known, we can derive the need for our own mission planning/scheduling software. Software for interpreting and distributing data will also be required, and should be agnostic wrt. the operations software. 
Some partner universities have software to consider. 

\subsection{Link Budget}

The link budget for this mission has been estimated considering an orbit height of 500 km and ground elevation angle $\epsilon = 10^{\circ}$ for each ground station. This preliminary study has been based on the data provided by GOMspace for the space segment and by KSAT and ISIS for the ground segment. In order to comply with frequency specifications from potential suppliers and partner stations, the downlink frequency carrier will be of about $\approx 2250$ MHz and the uplink of about $ \approx 2100$ MHz. In addition, to estimate the link budget it has been considered that the transmitter is formed by a power amplifier, some microwave circuits and an antenna; and the receiver is composed by a Low Noise Amplifier (LNA), some microwave circuits and the antenna. Figure \ref{fig:chain} shows the representation of the antenna chain for both TX and RX.

\begin{figure}[htbp]
	\centering
		\includegraphics[width=0.50\textwidth]{figs/chain_TX_RX.png}
	\caption{Block diagram of transmitter and receiving chain}
	\label{fig:chain}
\end{figure}

Regarding the satellite communication channel, this preliminary propagation losses take into account: free space loss, atmospheric gases loss, precipitation loss (considering 0.001 \% excess time in Svalbard climatic region) and a rough estimate of pointing, polarization and scintillation losses.

The preliminary link margin for a \hypso downlinking/uplinking to ground stations, such as KSAT, is shown in Table \ref{tab:link}. Options A, B, C, D are: S-band Downlink (GOMspace + KSAT), S-band Uplink (GOMspace + KSAT), S-band Downlink (GOMspace + ISIS), S-band Uplink (ISIS + GOMspace) respectively.



\begin{table}[htbp]
	\caption{Link Budget}
	\label{tab:link}
	\centering
		\begin{tabular}{|l|c|c|c|c|}
			\hline
			& A			&	 B	&	C	& D  \\ 
			\hline 															
		 TX & & & &\\
		\hline
		 Power input (W) & 0.48 & - & 26.80 & 46.99 \\
		Antenna gain (dB) & 7.90 & - & 7.90 & 33.28\\
		Losses (dB) & 1.00 & - & 1.00 & 1.00 \\
		EIRP (dBm) & 33.70 & 74.80 & 33.70 & 79.27 \\
		\hline
		Propagation Losses & & & & \\
		\hline
		Free space loss	(dB)	& 163.88	& 163.48	& 163.88  & 163.88\\
		Atmospheric loss (dB)	& 4.73 &	4.73	& 4.73 & 4.73 \\
		Other losses	(dB)	& 4.00 &	4.00 & 	4.00 & 4.00 \\
		Tot. propagation loss (dB) &	172.6	& 172.2	& 172.6 & 172.60 \\
			\hline
			RX & & & & \\
			\hline
			Antenna gain	(dB)	& 30.00 & 7.70 & 33.28 & 7.70\\
			Losses	(dB) & 1.00 & 1.00	& 1.00 & 1.00\\
			Noise figure	(dB) &	3.77	& 24.19	& 8.32 & 24.19\\
			Bandwidth	(MHz) &	2 &	2 &	2 & 2 \\
			\hline
			Link Quality & & & & \\
			\hline
			Carrier power	(dBm)& -109.91 & -90.70 &	-106.63 & -86.64\\
			Noise power	(dBm) &	-131.82 &	-111.40	& -127.27 & -111.40\\
			Signal to Noise ratio	(dB)	& 21.91	& 20.69	& 20.64 & 24.75 \\
			Margin	(dB)	& 14.43	& 13.20	& 13.15 & 17.27\\
			\hline
		\end{tabular}
\end{table}
%\red SUGGEST REMOVE:
%The Norwegian company Statsat are doing operations of the AIS-satellites and NORSat satellites. They now have four satellites operational, and a staff of only four-five people doing everything. They are in the process of re-writing their software based on the 5-6 years of experience they now have, in order to further simplify and stream-line operations. They are delivering an operational service for Kystverket, in addition to more ad-hoc support to research teams with other payloads on the NORSats. It could be very beneficial to try partner with them on mission software and control room ops. 
\black
